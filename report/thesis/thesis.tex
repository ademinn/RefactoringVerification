\documentclass{itmo-student-thesis}
\usepackage{placeins}
\usepackage{tikz}
\usetikzlibrary{trees}
\usetikzlibrary{arrows}
%% Указываем файл с библиографией.
\addbibresource{thesis.bib}
\renewcommand{\lstlistingname}{Листинг}

\begin{document}
\lstset{
    % language=Java,
    % basicstyle=\small\ttfamily,
    frame=single,
    % captionpos=b
}

\studygroup{6538}
\title{Статическая верификация рефакторных преобразований структуры программ на основе правил редукции в языке программирования Featherweight Java}
\author{Демин А. В.}
\supervisor{Баранов С. Н.}
\supervisordegree{докт. ф.-м. наук, профессор}
\publishyear{2015}

%% Эта команда генерирует титульный лист и аннотацию.
\makemastertitle

%% Оглавление
\tableofcontents

%% Макрос для введения. Совместим со старым стилевиком.
\startprefacepage
% \chapter{Введение}

В настоящее время рефакторинг~\cite{refactoring} является неотъемлемой частью процесса разработки программного обеспечения.
Рефакторинг проводят с целью улучшения читаемости кода и изменения его структуры для упрощения дальнейшего расширения.
Для проверки корректности рефакторных преобразований используются модульные тесты~\cite{tdd},
которые необходимо запускать после каждого преобразования.
Данный метод имеет свои недостатки -- в процессе тестирования можно лишь найти ошибки, но не доказать их отсутствие. 
Кроме того, требуется тратить время на написание и поддержку тестов.

Формальная верификация -- формальное доказательство соотетствия или несоответствия формального предмета верификации его формальному описанию.
Формальная верификация, как правило, используется при проверке ПО, к качеству которого предъявляются высокие требования (life-critical system).
Примерами такого ПО являются ПО медицинского оборудования.
Часто для формальной верификации используется статический анализ кода -- анализ, проводимый без реального выполнения программы.
Однако, как правило, статический анализ используется только для проверки самого исходного кода, а не изменений, вносимых в него в процессе разработки.

В настоящей работе предложен метод статической верификации рефакторных преобразований структуры программ.
Верификация проводится для программ на языке программирования Featherweight Java~\cite{fj}.
Для FJ определены правила редукции для всех типов выражений, что позволяет в процессе верификации вносить изменения в программы, не меняя их поведеия.
Так как FJ -- чистый язык программирования, то результаты, полученные в данной работе,
могут быть применены и к другим чистым языкам.
\chapter{Обзор предметной области}
\section{Статический анализ кода}
Статический анализ кода -- анализ программного обеспечения, проводимый без рельного выполнения исследуемых программ.
Статические анализаторы существуют для многих популярных языков программирования.
Примерами могут служить PVS-Studio для языка C++ или FindBugs для языка Java.

Статические анализаторы широко применяются в тех областях, где предъявляются высокие требования к качеству программного обеспечения.
Например, статический анализ используется для разработки авиационного ПО~\cite{static-air} или ПО для медицинского оборудования~\cite{static-med}.


Как правило, анализ проводится над какой-либо версией исходного кода,
и статическому анализатору прежде всего требуется построить и проверить корректность абстрактного синтаксического дерева.
Для этого статичесий анализатор либо содержит в себе лесический, синтаксический и семантический анализаторы,
либо, если это возможно, использует один из существующих компиляторов языка, на котором написана исследуемая программа.

Как правило, статический анализ используется для проверки некоторой фиксированной версии исходного кода.
В данной работе предлагается проверять не сам исходный код, а изменения, вносимые в него,
а именно -- рефакторные преобразования, то есть изменения, не затрагивающие поведение программы.
\section{Необходимые элементы теории формальных языков}
В данном разделе приведены определения, необходимые для описания синтаксического анализатора -- одного из компонентов статического анализатора.
Подробнее о теории формальных языков можно узнать в книге~\cite{hmu}.
\subsection{Формальные языки}
\begin{definition}
Алфавит -- конечное непустое множество символов.
\end{definition}
Далее для обозначения алфавита будет использоваться символ $\Sigma$.

Примером простейшего алфавита является бинарный алфавит $\Sigma = \{0, 1\}$.
Еще одним примером является алфавит, состоящий из цифр, круглых скобок и знаков $+, *$ -- алфавит языка арифметических выражений.
\begin{definition}
Слово или цепочка -- конечная последовательность символов некоторого алфавита.
\end{definition}
\begin{definition}
Длина цепочки -- число символов в цепочке.
\end{definition}
Для обозначения цепочки нулевой длины принято использовать греческую букву $\varepsilon$.

Последовательность $0101$ является словом длины 4 над приведенным выше бинарным алфавитом,
а $2 * (3 + 3)$ -- слово длины 7 над алфавитом языка арифметических выражений.
\begin{definition}
$\Sigma^k$ -- множество цепочек длины $k$ над алфавитом $\Sigma$.
\end{definition}
\begin{definition}
$\Sigma^* = \bigcup\limits_{k=0}^\infty$ -- множество всех цепочек над алфавитом $\Sigma$.
\end{definition}
\begin{definition}
Формальный язык над алфавитом $\Sigma$ -- некоторое подмножество $\Sigma^*$.
\end{definition}
В качестве примера формального языка можно привести язык палиндромов -- слов, которые читаются одинаково как слева направо, так и справа налево.
Палиндромами над бинарным алфавитом будут строки $010$ и $1001$.

Заметим, что любой язык программирования является формальным языком, а корректная программа на нем -- словом этого языка.
\subsection{Контекстно-свободные грамматики}
Контекстно-свободная грамматика состоит из следующих компонентов:
\begin{itemize}
    \item алфавит, элементы которого называют терминалами;
    \item конечное множество переменных, называемых нетерминалами; каждый нетерминал представляет формальный язык;
    \item стартовый символ грамматики -- один из нетерминалов, представляющий определямый язык;
    \item конечное множество правил вывода, или продукций, представляющих рекурсивное определение языка.
    Правило вывода представляет из себя пару из нетерминала, называемого головой продукции, и конечной цепочкой,
    состоящей из терминалов и нетерминалов, называемой телом продукции.
\end{itemize}
Формальное определение контекстно-свободной грамматики выглядит следующим образом.
\begin{definition}
Контекстно-свободная грамматика $G$ -- четверка $(\Sigma, N, S \in N, P \subset N \times (\Sigma \cup N)^*)$,
где $\Sigma$ -- алфавит, $N$ - множество нетерминалов, $S$ -- стартовый нетерминал, $P$ -- множество правил вывода.
\end{definition}
Типичным примером контекстно-свободной грамматики является грамматика арифметических выражений, представленная на рисунке~\ref{cf-expr}.
\begin{figure}[htb]
    \begin{align*}
        E &\rightarrow E + E\\
        E &\rightarrow T\\
        T &\rightarrow T * T\\
        T &\rightarrow F\\
        F &\rightarrow (E)\\
        F &\rightarrow n
    \end{align*}
    \caption{Конекстно-свободная грамматика языка арифметических выражений}
    \label{cf-expr}
\end{figure}
Для сокращения записи правила с одинаковой головой продукции часто объединяют. На рисунке~\ref{cf-expr-small} представлена сокращенная запись грамматики арифметических выражений.
\begin{figure}[htb]
    \begin{align*}
        E &\rightarrow E + E\ |\ T\\
        T &\rightarrow T * T\ |\ F\\
        F &\rightarrow (E)\ |\ n
    \end{align*}
    \caption{Сокращенная запись конекстно-свободной грамматики языка арифметических выражений}
    \label{cf-expr-small}
\end{figure}
\section{Необходимые элементы теории построения компиляторов}
В данном разделе приведено описание компонентов компилятора, необходимых для построения синтаксического анализатора.
Подробнее о построении компиляторов можно узнать в книге~\cite{dragon}.
\subsection{Лексический анализатор}
\subsection{Синтаксический анализатор}
\section{Редукция}

\section{Featherweight Java}
Featherweigth Java -- чистый язык программирования,
который является минимальным ядром для моделирования системы типов языка Java.
\subsection{Синтаксис Featherweight Java}
На рисунке~\ref{fj-syntax} приведен синтаксис определений классов, конструкторов, методов,
а также синтаксис выражений языка FJ. Здесь и далее будут использоваться следующие обозначения:
\begin{itemize}
    \item $A, B, C, D, E$ -- названия классов;
    \item $f, g$ -- названия полей классов;
    \item $m$ -- названия методов;
    \item $x$ -- переменные;
    \item $d, e$ -- выражения;
    \item $L$ -- определения классов;
    \item $K$ -- определения конструкторов;
    \item $M$ -- определения методов.
\end{itemize}
Каждый метод содержит неявную переменную $this$, и данное имя не может присутствовать в списке параметров.

Надчеркивание используется для обозначения последовательностей, возможно, пустых.
Так, $\overline{f}$ обозначает $f_1, \ldots, f_n$.
Аналогичным образом расшифровываются $\overline{C}, \overline{x}, \overline{e}$ и~т.\,д.
Через $\overline{C}\ \overline{f}$ будет обозначаться последовательность $C_1 f_1, \ldots, C_n, f_n$,
а через $this.\overline{f} = \overline{f}$ будет обозначаться $this.f_1 = f_1, \ldots, this.f_n = f_n$.
\begin{figure}[htb]
    \begin{align*}
        L &::= class\ C\ extends\ C\ \{\overline{C}\ \overline{f};\ K\ \overline{M}\}\\
        K &::= C(\overline{C}\ \overline{f})\{\ super(\overline{f});\ this.\overline{f} = \overline{f};\ \}\\
        M &::= C\ m(\overline{C}\ \overline{x}\{\ return\ e;\ \}\\
        e &::= x\ |\ e.f\ |\ e.m(\overline{e})\ |\ new\ C(\overline{e})\ |\ (C)e
    \end{align*}
    \caption{Сокращенное описание синтаксиса языка Featherweight Java}
    \label{fj-syntax}
\end{figure}
\subsection{Правила редукции}
Отношение редукции будем обозначать $e \rightarrow e'$, читается как <<выражение $e$ редуцируется в выражение $e'$ за один шаг>>.
Символом $\rightarrow^*$ будем обозначать рефлексивное и транзитивное замыкание отношения редукции.

Правила редукции выглядят следующим образом:
\begin{enumerate}
    \item $\cfrac{fields(C) = \overline{C}\overline{f}}{(new\ C(\overline{e})).f_i \rightarrow e_i}$;
    \item $\cfrac{mbody(m, C) = \overline{x}.e_0}{(new\ C(\overline{e})).m(\overline{d}) \rightarrow [\overline{d}/\overline{x}, new\ C(\overline{e})/this]e_0}$;
    \item $\cfrac{C <: D}{(D)(new\ C(\overline{e})) \rightarrow new\ C(\overline{e})}$;
    \item $\cfrac{e_0 \rightarrow e_0'}{e_0.f \rightarrow e_0'.f}$;
    \item $\cfrac{e_0 \rightarrow e_0'}{e_0.m(\overline{e}) \rightarrow e_0'.m(\overline{e})}$;
    \item $\cfrac{e_i \rightarrow e_i'}{e_0.m(\ldots, e_i, \ldots) \rightarrow e_0.m(\ldots, e_i, \ldots)}$;
    \item $\cfrac{e_i \rightarrow e_i'}{new\ C(\ldots, e_i, \ldots) \rightarrow new\ C(\ldots, e_i, \ldots)}$;
    \item $\cfrac{e_0 \rightarrow e_0'}{(C)e_0 \rightarrow (C)e_0'}$.
\end{enumerate}
Есть три правила редукции для базовых операций -- доступа к полю класса, вызова метода и приведения типов.
Через $[e/y]e_0$ обозначается результат подстановки выражения $e$ вместо переменной $y$ в выражении $e_0$.
Через $[\overline{d}/\overline{x}]e_0$ обозначается подстановка списка выражений вместо списка переменных.
\section{Рефакторинг}
У термина <<рефакторинг>> существует два определения -- рефакторинг как изменение в коде и рефакторинг как процесс внесения таких изменений.
\begin{definition}
Рефакторинг или рефакторное преобразование -- изменение во внутренней структуре программного обеспечения,
имеющее целью облегчить понимание его работы и упростить модификацию, не затрагивая наблюдаемого поведения.
\end{definition}
\begin{definition}
Производить рефакторинг -- изменять структуру программного обеспечения, применяя
ряд рефакторных преобразований, не затрагивая его поведения.
\end{definition}
Можно выделить несколько основных целей, с которыми проводится рефакторинг:
\begin{itemize}
    \item улучшение композиции ПО -- после проведения рефакторинга улучшается структура кода,
    становится проще вносить изменения в существующий код и добавлять новую функциональность;
    \item облегчение понимания ПО -- в процессе модификации незнакомого кода проще разобраться в деталях реализации;
    \item поиск ошибок -- при рефакторинге требуется глубоко вникать в модифицируемый код, что позволяет обнаружить ошибки;
    кроме того, после прояснения структуры программы некоторые ошибки становятся очевидными;
    \item ускорение разработки -- рефакторинг помогает сохранять хороший дизайн разрабатываемого ПО,
    что положительно сказывается на скорости разработки.
\end{itemize}
\subsection{Методы рефакторинга}
Рефакторное преобразование в общем смысле -- любое преобразование кода, не меняющее его поведения.
Однако существуют шаблонные преобразования, которые встречаются очень часто.
Такие шаблонные преобразования называют методами рефакторинга.
Ниже приводится несколько примеров наиболее часто встречающихся методов рефакторинга.
\subsubsection{Выделение метода}
Выделение метода -- один из наиболее часто проводимых методов рефакторинга.
При выделении метода фрагмент кода, который можно сгруппировать, преобразуется в метод,
название которого обозначает его назначение.

Выделение метода служит сразу двум целям:
\begin{itemize}
    \item избавление от дублирования кода -- при вынесении метода часто выносится код, который используется сразу в нескольких местах;
    кроме того, если выделен мелкий метод, повышается вероятность его использования другими методами;
    \item улучшение читаемости -- название метода служит цели документирования кода,
    и более длинные методы начинают выглядеть, как ряд комментариев.
\end{itemize}
\lstset{
    language=Java,
    basicstyle=\small\ttfamily,
    frame=single,
    captionpos=b
}
\section*{Выводы по главе 1}
\addcontentsline{toc}{section}{Выводы по главе 1}

\chapter{Описание предлагаемого подхода}

По определению, при проведении рефакторинга поведение программы не должно изменяться.
Поэтому, прежде чем описать алгоритмы верификации рефакторингов, требуется формализовать понятие <<одинаковое поведение>>.
Для этого введем понятие эквивалентности по редукции.
\begin{definition}
Выражения $\Gamma \vdash e_1$ и $\Gamma \vdash e_2$ назовем эквивалентными по редукции,
если для любых значений переменных из $\Gamma$ либо существует выражение $e'$ такое, что $e_1 \rightarrow^* e'$ и $e_2 \rightarrow^* e'$,
либо ни $e_1$ ни $e_2$ не могут быть редуцированы до нормальной формы.
Эквивалентные по редукции выражения будем обозначать $e_1 \equiv_r e_2$.
\end{definition}

\section{Выделение метода}

По определению, при выделении метода фрагмент кода, который можно сгруппировать, преобразуется в метод.
Выделенный метод должен находиться в том же классе, где первоначально находился выделенный фрагмент.
Если выделенный метод должен находиться в другом классе, то далее следует воспользоваться
другим методом рефакторинга -- перемещением метода.

При выделении метода и при его верификации следует учесть следующие детали:
\begin{itemize}
    \item тело выделенного метода не может содержать прямой рекурсии -- до проведения рефакторинга выделяемый метод еще не существовал, а значит выделенный фрагмент не мог содержать его вызовов;
    \item в родительских классах не должен быть определен метод с тем же именем -- в противном случае изменится поведение класса, в котором производится выделение метода;
    \item в дочерних классах не должен быть определен метод с тем же именем -- в противном случае в дочерних классах изменится поведение методов, в которых используется выделяемый метод.
\end{itemize}

Для начала определим внутреннюю редукцию по методу,
которая представляет собой подстановку тела метода при его вызове в другом методе,
определенном в том же классе или одном из его потомков.
\begin{definition}
Пусть в классе $C$ определен метод $m$, $mbody(m, C) = \overline{x}.e_m$.
Отношением редукции по методу $m$ назовем отношение со следующими правилами:
\begin{enumerate}
    \item $\cfrac{\Gamma \vdash this : D \qquad D <: C \qquad \overline{e} \rightarrow_m \overline{e}'}{this.m(\overline{e}) \rightarrow_m [\overline{e}'/\overline{x}]e_m}$
    \item $\cfrac{e \rightarrow_m e'}{e.f \rightarrow_m e'.f}$
    \item $\cfrac{e \rightarrow_m e' \qquad \overline{e} \rightarrow_m \overline{e}'}{e.m'(\overline{e}) \rightarrow_m e'.m'(\overline{e}')}$
    \item $\cfrac{\overline{e} \rightarrow_m \overline{e}'}{new\ C(\overline{e}) \rightarrow_m new\ C(\overline{e}')}$
    \item $\cfrac{e \rightarrow_m e'}{(D)e \rightarrow_m (D)e'}$
\end{enumerate}
\end{definition}
Заметим, что подстановка будет происходить, только если редукция проводится для тела одного из методов класса $C$ или его потомков.
В противном случае, для выражения $e$ существует только одно выражение, с которым оно состоит в отношении редукции по методу $m$ -- само выражение $e$.

При помощи редукции по методу введем понятие внутренней подстановки, которое необходимо для доказательства корректности алгоритма верификации выделения метода.
\begin{definition}
Пусть в классе $C$ определены методы $m$ и $m_0$, $mbody(m_0, C) = \overline{x}.e_0$.
Пусть после внесения изменений метод $m_0$ преобразовался в метод $m_0'$, $mbody(m_0', C) = \overline{x}.e_0'$, причем $e_0 \rightarrow_m e_0'$.
Такое преобразование будем называть внутренней подстановкой и обозначать $m_0' = inner_m(m_0)$.
\end{definition}

\begin{theorem}
Пусть в классе $C$ определены методы $m$ и $m_0$ и метод $m$ не содержит прямой рекурсии. Пусть $m_0' = inner_m(m_0)$.
Тогда, $\forall e : C, \overline{e}$ верно: $e.m_0(\overline{e}) \equiv_r e.m_0'(\overline{e})$.
\end{theorem}
\begin{proof}
Пусть $e$ не может быть редуцирован к виду $new\ C(\overline{e}')$. Тогда не могут быть редуцированы вызовы методов $m_0$ и $m_0'$, а значит оба выражения не могут быть редуцированы до нормальной формы.

Пусть $e$ может быть редуцирован к виду $new\ C(\overline{e}')$.

Проведем редукцию вызова метода $m_0$ по второму правилу редукции, затем в полученном выражении проведем редукцию всех вызовов методов $m$ тоже по второму правилу
(это можно сделать, так как на предыдущем шаге была произведена подстановка выражения $new\ C(\overline{e}')$ вместо переменной $this$). Полученное выражение обозначим $e_0$.

Проведем редукцию вызова метода $m_0'$ по второму правилу редукции, обозначим полученное выражение $e_0'$.

Из определения редукции по методу следует, что выражения $e_0$ и $e_0'$ равны.
В самом деле, процесс их вывода отличается лишь порядком проведения подстановок -- для $m_0$ сначала вместо $this$ было подставлено $new\ C(\overline{e}')$, а затем раскрыт вызов метода $m$;
для $m_0'$ сначала был раскрыт вызов метода $m$ (из определения внутренней подстановки), а затем вместо $this$ было подставлено $new\ C(\overline{e}')$.

Значит $e.m_0(\overline{e}) \equiv_r e.m_0'(\overline{e})$.
\end{proof}

Таким образом, алгоритм верификации выделения метода выглядит следующим образом.
\begin{enumerate}
    \item Проверить, что выделенный метод не содержит прямой рекурсии.
    \item Проверить, что родительские классы не содержат метод с таким же именем.
    \item Проверить, что дочерние классы не содержат метод с таким же именем.
    \item Для всех методов, содержащих вызовы выделенного метода, проверить, что тело метода до и после рефакторинга находятся в отношении редукции по методу. Проверка осуществляется конструктивным методом.
\end{enumerate}

\section{Замена условного оператора полиморфизмом}
Согласно определению рассматриваемого метода рефакторинга, происходит преобразование условного оператора, поведение которого зависит от типа объекта (на практике чаще всего используется оператор switch).
Каждая ветвь условного оператора перемещается в перегруженный метод подкласса, а метод базового класса объявляется абстрактным.

\subsection{Ошибки времени выполнения}
Заметим, что в языке Featherweight Java нет понятия <<абстрактный метод>>. Для обозначения того, что метод является абстрактным, вызов такого метода должен приводить к ошибке выполнения.
Такого поведения можно добиться, например, используя приведение типов. В листинге~\ref{exception} приведено определение класса, использующегося для вызова ошибок выполнения.
\begin{lstlisting}[float=htb,label=exception,caption=Определение класса Exception]
class Exception extends Object {
    Exception() {
        super();
    }

    Exception throw() {
        return (Exception) new Object();
    }
}
\end{lstlisting}

\subsection{Реализация switch-выражения}
В языке Featherweight Java нет примитивного логического типа данных, и соответственно нет условных операторов. Однако, их можно реализовать средствами самого языка, как это делается в лямбда-исчислении.
В листинге~\ref{bool} привден пример реализации логического типа данных. Заметим, что метод $choose$ представляет собой условное выражение, и, по сути, заменяет конструкцию $if-then-else$,
использующуюся в других языках.

\begin{lstlisting}[float=htb,label=bool,caption=Определение логического типа данных]
class Bool extends Object {
    Bool() { super(); }

    Object choose(Object trueExpr, Object falseExpr) {
        return new Exception().throw();
    }

    Bool and(Bool value) {
        return (Bool) new Exception().throw();
    }

    Bool or(Bool value) {
        return (Bool) new Exception().throw();
    }

    Bool not() {
        return (Bool) new Exception().throw();
    }

    Bool equals(Bool value) {
        return this.and(value).or(this.not().and(value.not()));
    }
}

class True extends Bool {
    True() { super(); }

    Object choose(Object trueExpr, Object falseExpr) {
        return trueExpr;
    }

    Bool and(Bool value) {
        return value;
    }

    Bool or(Bool value) {
        return new True();
    }

    Bool not() {
        return new False();
    }
}

class False extends Bool {
    False() { super(); }

    Object choose(Object trueExpr, Object falseExpr) {
        return falseExpr;
    }

    Bool and(Bool value) {
        return new False();
    }

    Bool or(Bool value) {
        return value;
    }

    Bool not() {
        return new True();
    }
}
\end{lstlisting}
Теперь перейдем к определению switch-выражений. Любое switch-выражение представляет собой список пар, состоящих из логического выражения и выражения, представляющего результат.
При вычислении по-очереди проверяются все логические выражения. Результатом будет выражение, соответствующего тому условию, которое оказалось верным и находилось в списке раньше остальных.
Таким образом, для реализации switch-выражений требуется сначала реализовать список пар условие-результат. В листинге~\ref{switch-nodes} приведена реализация элементов такого списка.
\begin{lstlisting}[float=htb,label=switch-nodes,caption=Реализация элементов списка для switch-выражений]
class SwitchNode extends Object {
    SwitchNode() {
        super();
    }

    Bool isLeaf() {
        return (Bool) new Exception().throw();
    }
}

class SwitchLeafNode extends SwitchNode {
    SwitchLeafNode() {
        super();
    }

    Bool isLeaf() {
        return new True();
    }
}

class SwitchInnerNode extends SwitchNode {
    Bool guard;
    Object value;
    SwitchNode prev;

    SwitchInnerNode(Bool guard, Object value, SwitchNode prev) {
        super();
        this.guard = guard;
        this.value = value;
        this.prev = prev;
    }

    Bool isLeaf() {
        return new False();
    }
}
\end{lstlisting}

В листинге~\ref{switch} приведена реализация самого списка и реализация switch-выражений. Так как при приведенной реализации связный список представляет собой стек,
а на практике удобнее использовать очередь, то необходим метод, разворачивающий имеющийся список в обратном порядке.
Данную реализацию можно было бы обобщить, используя структуры данных $List$ и $Pair$, однако,
так как в языке Featherweigth Java отсутствует такая конструкция, как generics~\cite{java},
то это привело бы к большому числу операторов приведения типа.
\begin{lstlisting}[float=htb,label=switch,caption=Определение switch-выражений.]
class SwitchList extends Object {
    SwitchNode top;
    SwitchList(SwitchNode top) {
        super();
        this.top = top;
    }

    SwitchList append(Bool guard, Object value) {
        return new SwitchList(new SwitchInnerNode(guard, value, this.top));
    }

    SwitchList appendAllInnerNode(SwitchInnerNode node) {
        return new SwitchList(new SwitchInnerNode(
            node.guard, node.value, this.top)).appendAll(node.prev);
    }

    SwitchList appendAll(SwitchNode node) {
        return node.isLeaf().choose(
            this,
            this.appendAllInnerNode((SwitchInnerNode) node));
    }

    SwitchList revert() {
        return new SwitchList(new SwitchLeafNode()).appendAll(this.top);
    }

    Bool isEmpty() {
        return this.top.isLeaf();
    }

    SwitchInnerNode getInnerTop() {
        return (SwitchInnerNode) this.top;
    }

    Bool getTopGuard() {
        return this.getInnerTop().guard;
    }

    Object getTopValue() {
        return this.getInnerTop().value;
    }

    SwitchList remove() {
        return new SwitchList(this.getInnerTop().prev);
    }
}
class Switch extends Object {
    Switch() { super(); }

    Object switchList(SwitchList list) {
        return list.isEmpty().choose(
            new Exception().throw(),
            list.getTopGuard().choose(
                list.getTopValue(),
                this.switchList(list.remove())));
    }

    Object switch(SwitchList list) {
        return this.switchList(list.revert());
    }
}
\end{lstlisting}

\subsection{Алгоритм верификации}
Перейдем к формализации рассматриваемого метода рефакторинга. Прежде чем применять <<замену условного оператора полиморфизмом>>,
следует создать необходимую иерархию наследования. Кроме того, если условное выражение является частью более крупного метода,
то его следует вынести в отдельный метод, используя <<выделение метода>>.

Таким образом, непосредственно перед применением рефакторинга имеется класс $C$, содержащий метод $m$, и классы $D_1, \ldots, D_n$, являющиеся дочерними классами класса $C$.
Тело метода $m$ представляет собой switch-выражение, результат которого зависит от типа объекта.
Должны выполняться два требования:
\begin{itemize}
    \item число ветвей в switch-выражении должно совпадать с числом дочерних классов класса $C$;
    \item все условные выражения в switch-выражении должны приводиться к нормальной форме не более чем за константное число шагов.
\end{itemize}
Исходя из этих требований будем считать, что switch-выражение содержит $n$ пар $(guard_1, result_1), \ldots, (guard_n, result_n)$,
причем для любых $1 \le i, j \le n$ выражение $[new\ D_i(\overline{e})/this]guard_j$ приводится к нормальной форме за константное число шагов.

При проведении верификации прежде всего требуется установить взаимно-однозначное соответствие между ветвями switch-выражения и дочерними классами класса $C$.
Для этого вычисляются значения выражений $[new\ D_i(\overline{e})/this]guard_j$ для всех дочерних классов и для всех условий switch-выражения.
Далее классу $D_i$ ставится в соответствие выражение $guard_j$ такое, что $[new\ D_i(\overline{e})/this]guard_j \rightarrow^* True$,
и для всех $1 \le k < j: [new\ D_i(\overline{e})/this]guard_k \rightarrow^* False$. Если одна и та же ветвь switch-выражения соответствует более чем одному
дочернему классу, то для такого выражения провести рефакторинг невозможно.

Далее остается лишь проверить, что в каждом из дочерних классов после проведения рефакторинга появился метод $m$,
тело которого совпадает с результатом ветви switch-выражения, поставленной в соответствие классу на предыдущем шаге.

Такое преобразование является рефакторингом, так как для любого $i$ верно $new\ D_i(\overline{e_c}).m(\overline{e_m}) \equiv_r new\ D_i'(\overline{e_c}).m(\overline{e_m})$,
где $D_i$ и $D_i'$ -- дочерние классы до и после рефакторинга. В самом деле, все условные выражения приводятся к нормальной форме и результат первой ветви,
для которой условное выражение является истинным, совпадает с телом метода после рефакторинга.

Таким образом, алгоритм верификации замены условного оператора полиморфизмом выглядит следующим образом.
\begin{enumerate}
    \item Проверить, что тело преобразуемого метода представляет собой switch-выражение.
    \item Проверить, что число ветвей в switch-выражении совпадает с числом дочерних классов.
    \item Для каждого дочернего класса вычислить значения условных выражений всех ветвей switch-выражения.
    \item Каждому дочернему классу поставить в соответствие первую ветвь switch-выражения, условное выражение которой является истинным.
    \item Проверить, что каждая ветвь была поставлена в соответствие только одному классу.
    \item Для каждого дочернего класса проверить, что тело метода после рефакторинга совпадает результатом соответствующей ветви switch-выражения.
\end{enumerate}
\section*{Выводы по главе 2}
\addcontentsline{toc}{section}{Выводы по главе 2}
Описаны алгоритмы верификации двух методов рефакторинга -- <<выделение метода>> и <<замена условного оператора полиморфизмом>>.
Доказана корректность предложенных алгоритмов.
\chapter{Практическая реализация}
\section{Разбор Featherweight Java}
\subsection{Лексический анализатор}
Для разбиения исходного кода на токены использовался генератор лексических анализаторов Alex~\cite{alex}.
\subsection{Синтаксический анализатор}
\begin{figure}[H]
    \begin{align*}
        Classes \rightarrow&\ Class\ Classes\\
        Class \rightarrow&\ class\ ClassName\ extends\ ClassName\\
        &\ \{\ Fields\ Constructor\ Methods\ \}\\
        Fields \rightarrow&\ Field\ Fields\ |\ \varepsilon\\
        Field \rightarrow&\ ClassName\ FieldName;\\
        Constructor \rightarrow& \ ClassName\ (\ Parameters\ )\\
        &\ \{\ super\ (\ Expressions\ );\ Assigns\ \}\\
        Assigns \rightarrow&\ Assign\ Assigns\ |\ \varepsilon\\
        Assing \rightarrow&\ this\ .\ FieldName\ =\ VarName;\\
        Methods \rightarrow&\ Method\ Methods\ |\ \varepsilon\\
        Method \rightarrow&\ ClassName\ MethodName\ (\ Parameters\ )\\
        &\ \{\ return\ Expression;\ \}\\
        Expression \rightarrow&\ VarName\\
        &\ |\ Expression\ .\ FieldName\\
        &\ |\ Expression\ .\ MethodName\ (\ Expressions\ )\\
        &\ |\ new\ ClassName\ (\ Expressions\ )\\
        &\ |\ (\ ClassName\ )\ Expression\\
        Parameters \rightarrow&\ Parameter\ ,\ Parameters\ |\ Parameter\\
        Parameter \rightarrow&\ ClassName\ VarName\\
        Expressions \rightarrow&\ Expression\ ,\ Expressions\ |\ Expression\\
        ClassName \rightarrow&\ id\\
        FieldName \rightarrow&\ id\\
        MethdoName \rightarrow&\ id\\
        VarName \rightarrow&\ id
    \end{align*}
    \caption{Контекстно-свободная грамматика языка Featherweight Java}
    \label{cf-fj}
\end{figure}
Для построения дерева разбора использовался генератор разборщиков Happy~\cite{happy}.
\subsection{Семантический анализатор}

\section{Верификация выделения метода}
\chapter{Примеры практического использования}

\section{Выделение метода}
В листингах~\ref{pf-before} и~\ref{pf-after} приведен код до и после выделения метода $createPair$.
\begin{lstlisting}[float,label=pf-before,caption=Код до выделения метода]
class Pair extends Object {
    Object first;
    Object second;

    Pair(Object first, Object second) {
        super();
        this.first = first;
        this.second = second;
    }
}

class PairFactory extends Object {
    PairFactory() {
        super();
    }

    Pair createPairFst(Object fst) {
        return new Pair(fst, new Object());
    }

    Pair createPairSnd(Object snd) {
        return new Pair(new Object(), snd);
    }
}
\end{lstlisting}

\begin{lstlisting}[float,label=pf-after,caption=Код после выделения метода]
class PairFactory extends Object {
    PairFactory() {
        super();
    }

    Pair createPair(Object fst, Object snd) {
    	return new Pair(fst, snd);
    }

    Pair createPairFst(Object fst) {
        return this.createPair(fst, new Object());
    }

    Pair createPairSnd(Object snd) {
        return this.createPair(new Object(), snd);
    }
}
\end{lstlisting}


Процесс верификации.
\begin{enumerate}
    \item Метод $createPair$ не содержит прямой рекурсии.
    \item В классе $Object$ метод $createPair$ отсутствует.
    \item Дочерние классы отсутствуют.
    \item Вызов выделенного метода присутствует в методах $createPairFst$ и $createPairSnd$.
    \begin{itemize}
        \item Для метода $createPairFst$ выполняется
        $$this.createPair(fst,\ new\ Object()) \rightarrow_m new\ Pair(fst,\ new\ Object()).$$
        В данном случае была произведена подставновка
        $$[fst/fst,\ new\ Object()/snd]new\ Pair(fst,\ snd)$$
        вместо вызова метода $createPair$.
        \item Для метода $createPairSnd$ выполняется
        $$this.createPair(new\ Object(),\ snd) \rightarrow_m new\ Pair(new\ Object(),\ snd).$$
        В данном случае была произведена подставновка
        $$[new\ Object()/fst,\ snd/snd]new\ Pair(fst,\ snd)$$
        вместо вызова метода $createPair$.
    \end{itemize}
\end{enumerate}

\section{Замена условного оператора полиморфизмом}
Введем натуральные числа по аналогии с лямбда-исчислением.
\begin{lstlisting}[float,label=nat,caption=Определение натуральных чисел]
class Nat extends Object {
    Nat() {
       super();
    }

    Bool isZero() {
        return (Bool) new Exception().throw();
    }

    Nat add(Nat value)

    Bool equals(Nat value)
}

class Zero extends Nat {
    Zero() {
        super();
    }

    Bool isZero() {
        return True;
    }

    Nat add(Nat value) {
        return value;
    }

    Bool equals(Nat value) {
        return value.isZero();
    }
}

class Succ extends Nat {
    Nat prev;
    Succ(Nat prev) {
        super();
        this.prev = prev;
    }

    Bool isZero() {
        return False;
    }

    Nat add(Nat value) {
        return prev.add(Succ(value));
    }

    Bool equals(Nat value) {
        return value.isZero().choose(
            new False(),
            prev.equals(((Succ) value).prev));
    }
}
\end{lstlisting}

В качестве иллюстрации приведем классический пример из книги Фаулера <<Рефакторинг. Улучшение существующего кода>>.
В листингах~\ref{et-before} и~\ref{et-after} приведен код до и после замены switch-выражения в методе $payAmount$ полиморфизмом.

\begin{lstlisting}[float,label=et-before,caption=Код до рефакторинга]
class Employee extends Object {
    ...

    Nat getMonthlySalary() { ... }

    Nat getCommission() { ... }

    Nat getBonus() { ... }
}

class EmployeeType {
    ...

    Nat payAmount(Employee emp) {
        return (Nat) new Switch().switch(new SwitchList(new SwitchLeafNode())
            .append(
                this.getTypeCode().equals(0),
                emp.getMonthlySalary()),
            .append(
                this.getTypeCode().equals(1), 
                emp.getMonthlySalary().add(emp.getCommission())),
            .append(
                this.getTypeCode().equals(2),
                emp.getMonthlySalary().add(emp.getBonus())));
    }

    Nat getTypeCode() {
        return (Nat) new Exception().throw();        
    }
}

class Engineer extends EmployeeType {
    ...

    Nat getTypeCode() {
        return 0;
    }
}

class Salesman extends EmployeeType {
    ...

    Nat getTypeCode() {
        return 1;
    }
}

class Manager extends EmployeeType {
    ...

    Nat getTypeCode() {
        return 2;
    }
}
\end{lstlisting}

\begin{lstlisting}[float,label=et-after,caption=Код после рефакторинга]
class EmployeeType {
    ...

    Nat payAmount(Employee emp) {
        return (Nat) new Exception().throw()
    }
}

class Engineer extends EmployeeType {
    ...

    Nat payAmount(Employee emp) {
        return emp.getMonthlySalary();
    }
}

class Salesman extends EmployeeType {
    ...

    Nat payAmount(Employee emp) {
        return emp.getMonthlySalary().add(emp.getCommission());
    }
}

class Manager extends EmployeeType {
    ...

    Nat payAmount(Employee emp) {
        return emp.getMonthlySalary().add(emp.getBonus());
    }
}
\end{lstlisting}
Процесс верификации.

\begin{enumerate}
    \item Тело метода $payAmount$ представляет собой switch-выражение.
    \item Число ветвей в switch-выражении и число дочерних классов совпадают.
    \item Значения условных выражений для всех дочерних классов приведены в таблице~\ref{et-c}.
    \begin{table}[H]
    \begin{center}
    \begin{tabular}{|c|c|c|c|}
    \hline
    Номер ветви & 1 & 2 & 3\\
    \hline
    $Engineer$ & True & False & False\\
    \hline
    $Salesman$ & False & True & False\\
    \hline
    $Manager$ & False & False & True\\
    \hline
    \end{tabular}
    \end{center}
    \caption{Значения условных выражений для дочерних классов класса $EmployeeType$}
    \label{et-c}
    \end{table}
    \item Исходя из результатов предыдущего пункта, классу $Engineer$ ставится в соответствие первая ветвь switch-выражения, классу $Salesman$ -- вторая, $Manager$ -- третья.
    \item Каждая ветвь поставлена в соответствие только одному классу.
    \item Для всех дочерних классов тело метода после рефакторинга совпадает с результатом соответствующей ветви switch-выражения.
\end{enumerate}
\startconclusionpage
\printbibliography[heading=trueHeading]

\end{document}