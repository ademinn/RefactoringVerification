\documentclass[a4paper]{report}

%uncomment to see the references
%\usepackage{showkeys}
\usepackage[T2A]{fontenc}

\usepackage{algorithm}
\usepackage{algorithmic}
\usepackage[english,russian]{babel}
\usepackage{multirow}
\usepackage[backend=biber,sorting=none,sortcites=true,bibstyle=sty/gost71,maxnames=99,citestyle=numeric-comp,babel=other]{biblatex}

\defbibenvironment{bibliography}
  {\list
     {\printfield[labelnumberwidth]{labelnumber}.}
     {\setlength{\labelwidth}{2\labelnumberwidth}%
      \setlength{\leftmargin}{\labelwidth}%
      \setlength{\labelsep}{\biblabelsep}%
      \addtolength{\leftmargin}{\labelsep}%
      \setlength{\itemsep}{\bibitemsep}%
      \setlength{\parsep}{\bibparsep}}%
      \renewcommand*{\makelabel}[1]{\hss##1}}
  {\endlist}
  {\item}

\usepackage[utf8]{inputenc}
\usepackage{csquotes}
%\usepackage{expdlist}
%\usepackage[nottoc,notbib]{tocbibind}
\usepackage[pdftex]{graphicx}
\graphicspath{{pic/}}
\usepackage{amsmath}
\usepackage{amssymb}
\usepackage{amsthm}
\usepackage{amsfonts}
\usepackage{amsxtra}
\usepackage{sty/dbl12}
%\usepackage{srcltx}
\usepackage{epsfig}
%\usepackage{verbatim}
\usepackage{sty/rac}
\usepackage{color}
\usepackage{listings}
\usepackage[singlelinecheck=false]{caption}
\usepackage{calc}
\usepackage{tikz}
\usetikzlibrary{calc,intersections,through,backgrounds,chains,positioning,fit}
\usepackage{xcolor, colortbl}
\usepackage{varwidth}
\definecolor{light-gray}{RGB}{230,230,230}
% \bibliographystyle{sty/gost71s}

%%%%%%%%%%%%%%%%%%%%%%%%%%%%%%%%%%%%%%%%%%%%%%%%%%%%%%%%%%%%%%%%%%%%%%%%%%%%%%

\captionsetup[figure]{justification=centering,   position=bottom, skip=0pt}
\captionsetup[table] {justification=raggedright, position=top,    skip=0pt}

% Redefine margins and other page formatting

\setlength{\oddsidemargin}{0.5in}

% Various theorem environments. All of the following have the same numbering
% system as theorem.

\theoremstyle{plain}
\newtheorem{theorem}{Теорема}
\newtheorem{prop}[theorem]{Утверждение}
\newtheorem{corollary}[theorem]{Следствие}
\newtheorem{lemma}[theorem]{Лемма}
\newtheorem{question}[theorem]{Вопрос}
\newtheorem{conjecture}[theorem]{Гипотеза}
\newtheorem{assumption}[theorem]{Предположение}

\theoremstyle{definition}
\newtheorem{definition}[theorem]{Определение}
\newtheorem{notation}[theorem]{Обозначение}
\newtheorem{condition}[theorem]{Условие}
\newtheorem{example}[theorem]{Пример}
%\newtheorem{algorithm}[theorem]{Алгоритм}
\floatname{algorithm}{Листинг}
\renewcommand{\algorithmicrequire}{\textbf{Вход:}}

%\newtheorem{introduction}[theorem]{Introduction}

\renewcommand{\proof}{\textbf{Доказательство.}~}


%\def\startprog{\begin{lstlisting}[language=Java,basicstyle=\normalsize\ttfamily]}

%\theoremstyle{remark}
%\newtheorem{remark}[theorem]{Remark}
%\include{header}
%%%%%%%%%%%%%%%%%%%%%%%%%%%%%%%%%%%%%%%%%%%%%%%%%%%%%%%%%%%%%%%%%%%%%%%%%%%%%%%

\numberwithin{theorem}{chapter}        % Numbers theorems "x.y" where x
                                        % is the section number, y is the
                                        % theorem number

%\renewcommand{\thetheorem}{\arabic{chapter}.\arabic{theorem}}

%\makeatletter                          % This sequence of commands will
%\let\c@equation\c@theorem              % incorporate equation numbering
%\makeatother                           % into the theorem numbering scheme

%\renewcommand{\theenumi}{(\roman{enumi})}

%%%%%%%%%%%%%%%%%%%%%%%%%%%%%%%%%%%%%%%%%%%%%%%%%%%%%%%%%%%%%%%%%%%%%%%%%%%%%%%

\binoppenalty=10000
\relpenalty=10000

\addbibresource{thesis.bib}
% \bibliography{bib}

\begin{document}

% Begin the front matter as required by Rackham dissertation guidelines
\initializefrontsections

\pagestyle{title}
\begin{center}
Санкт-Петербургский национальный исследовательский университет \\ информационных технологий, механики и оптики

\vspace{2cm}

Факультет информационных технологий и программирования

Кафедра компьютерных технологий

\vspace{3cm}

{\Large Демин Андрей Васильевич}

\vspace{2cm}

\vbox{\LARGE\bfseries
Статическая верификация рефакторных преобразований структуры программ на основе правил редукции в языке программирования Featherweight Java\\
}

\vspace{4cm}

{\Large Научный руководитель: доктор физико-математических наук, профессор С.~Н.~Баранов.\\
Соруководитель: кандидат технических наук С.~МакКивер}

\vspace{5cm}

Санкт-Петербург\\ 2015
\end{center}

\newpage

\setcounter{page}{3}
\pagestyle{plain}

\tableofcontents
%\listoffigures

% Chapters
\startthechapters
\startprefacepage
% \chapter{Введение}

В настоящее время рефакторинг~\cite{refactoring} является неотъемлемой частью процесса разработки программного обеспечения.
Рефакторинг проводят с целью улучшения читаемости кода и изменения его структуры для упрощения дальнейшего расширения.
Для проверки корректности рефакторных преобразований используются модульные тесты~\cite{tdd},
которые необходимо запускать после каждого преобразования.
Данный метод имеет свои недостатки -- в процессе тестирования можно лишь найти ошибки, но не доказать их отсутствие. 
Кроме того, требуется тратить время на написание и поддержку тестов.

Формальная верификация -- формальное доказательство соотетствия или несоответствия формального предмета верификации его формальному описанию.
Формальная верификация, как правило, используется при проверке ПО, к качеству которого предъявляются высокие требования (life-critical system).
Примерами такого ПО являются ПО медицинского оборудования.
Часто для формальной верификации используется статический анализ кода -- анализ, проводимый без реального выполнения программы.
Однако, как правило, статический анализ используется только для проверки самого исходного кода, а не изменений, вносимых в него в процессе разработки.

В настоящей работе предложен метод статической верификации рефакторных преобразований структуры программ.
Верификация проводится для программ на языке программирования Featherweight Java~\cite{fj}.
Для FJ определены правила редукции для всех типов выражений, что позволяет в процессе верификации вносить изменения в программы, не меняя их поведеия.
Так как FJ -- чистый функциональный язык программирования, то результаты, полученные в данной работе,
могут быть применены и к другим чистым функциональным языкам.
\chapter{Обзор предметной области}
\section{Рефакторинг}
У термина <<рефакторинг>> существует два определения -- рефакторинг как изменение в коде и рефакторинг как процесс внесения таких изменений.
\begin{definition}
Рефакторинг или рефакторное преобразование -- изменение во внутренней структуре программного обеспечения,
имеющее целью облегчить понимание его работы и упростить модификацию, не затрагивая наблюдаемого поведения.
\end{definition}
\begin{definition}
Производить рефакторинг -- изменять структуру программного обеспечения, применяя
ряд рефакторных преобразований, не затрагивая его поведения.
\end{definition}
Можно выделить несколько основных целей, с которыми проводится рефакторинг:
\begin{itemize}
    \item улучшение композиции ПО -- после проведения рефакторинга улучшается структура кода,
    становится проще вносить изменения в существующий код и добавлять новую функциональность;
    \item облегчение понимания ПО -- в процессе модификации незнакомого кода проще разобраться в деталях реализации;
    \item поиск ошибок -- при рефакторинге требуется глубоко вникать в модифицируемый код, что позволяет обнаружить ошибки;
    кроме того, после прояснения структуры программы некоторые ошибки становятся очевидными;
    \item ускорение разработки -- рефакторинг помогает сохранять хороший дизайн разрабатываемого ПО,
    что положительно сказывается на скорости разработки.
\end{itemize}
\subsection{Методы рефакторинга}
Рефакторное преобразование в общем смысле -- любое преобразование кода, не меняющее его поведения.
Однако существуют шаблонные преобразования, которые встречаются очень часто.
Такие шаблонные преобразования называют методами рефакторинга.
Ниже приводится несколько примеров наиболее часто встречающихся методов рефакторинга.
\subsubsection{Выделение метода}
Выделение метода -- один из наиболее часто проводимых методов рефакторинга.
При выделении метода фрагмент кода, который можно сгруппировать, преобразуется в метод,
название которого обозначает его назначение.

Выделение метода служит сразу двум целям:
\begin{itemize}
    \item избавление от дублирования кода -- при вынесении метода часто выносится код, который используется сразу в нескольких местах;
    кроме того, если выделен мелкий метод, повышается вероятность его использования другими методами;
    \item улучшение читаемости -- название метода служит цели документирования кода,
    и более длинные методы начинают выглядеть, как ряд комментариев.
\end{itemize}
\section{Существующие работы по теме}
Наиболее полное описание методов рефакторинга, способов их применения, цели проведения рефакторинга и его влияния на процесс разработки
приведено в книге Фаулера <<Рефакторинг. Улучшение существующего кода>>.
После ее выхода было предложено несколько способов формализации и верификации рефакторингов,
основанных на проверке инвариантов при внесении изменений~\cite{Rustan04objectinvariants, DBLP:journals/eceasst/MassoniGB06}.
Однако для их применения требуется в явном виде описывать инварианты при написании программы,
что не позволяет полностью автоматизировать процесс верификации.

Рефакторные преобразования архитектуры программ были формализованы для языков описания архитектуры,
таких как Wright~\cite{Allen98specifyingand} и Darwin~\cite{Ehrig:2006:FAG:1121741}.
В работе~\cite{Bisztray_verificationof} рассмотрена верификация архитектурных рефакторингов для моделей,
описанных на языке UML.

В работе~\cite{Taentzer:1998:DCM:645872.668868} вводится понятие <<distributed graph transformation>>,
с помощью которого архитектурные изменения в распределенных системах рассматриваются как
измененения на локальном уровне и на уровне системы в целом.

Еще один способ формализации архитектурных рефакторингов представлен в работе~\cite{Hirsch:1998:GGC:288408.288426}.
В ней система в целом представляется в виде гиперграфа, а вносимые изменения
представляют собой синхронную замену ребер.

Однако все эти подходы могут применяться только при перепроектировании общей архитектуры системы,
представленной на некотором языке высокого уровня, которая лишь описывает общую модель.
Поэтому, данные подходы неприменимы при написании непосредственно исходного кода разрабатываемого программного обеспечения.

\section{Статический анализ кода}
Статический анализ кода -- анализ программного обеспечения, проводимый без рельного выполнения исследуемых программ~\cite{Wichmann95industrialperspective}.
Статические анализаторы существуют для многих популярных языков программирования.
Примерами могут служить PVS-Studio~\cite{pvs-studio} для языка C++ или FindBugs~\cite{findbugs} для языка Java.

Статические анализаторы широко применяются в тех областях, где предъявляются высокие требования к качеству программного обеспечения.
Например, статический анализ используется для разработки авиационного ПО~\cite{static-air} или ПО для медицинского оборудования~\cite{static-med}.


Как правило, анализ проводится над какой-либо версией исходного кода,
и статическому анализатору прежде всего требуется построить и проверить корректность абстрактного синтаксического дерева.
Для этого статичесий анализатор либо содержит в себе лесический, синтаксический и семантический анализаторы,
либо, если это возможно, использует один из существующих компиляторов языка, на котором написана исследуемая программа.

Как правило, статический анализ используется для проверки некоторой фиксированной версии исходного кода.
В данной работе предлагается проверять не сам исходный код, а изменения, вносимые в него,
а именно -- рефакторные преобразования, то есть изменения, не затрагивающие поведение программы.
\section{Необходимые элементы теории формальных языков}
В данном разделе приведены определения, необходимые для описания синтаксического анализатора -- одного из компонентов статического анализатора.
Подробнее о теории формальных языков можно узнать в книге~\cite{hmu}.
\subsection{Формальные языки}
\begin{definition}
Алфавит -- конечное непустое множество символов.
\end{definition}
Далее для обозначения алфавита будет использоваться символ $\Sigma$.

Примером простейшего алфавита является бинарный алфавит $\Sigma = \{0, 1\}$.
Еще одним примером является алфавит, состоящий из цифр, круглых скобок и знаков <<$+$>>, <<$*$>> -- алфавит языка арифметических выражений.
\begin{definition}
Слово или цепочка -- конечная последовательность символов некоторого алфавита.
\end{definition}
\begin{definition}
Длина цепочки -- число символов в цепочке.
\end{definition}
Для обозначения цепочки нулевой длины принято использовать греческую букву $\varepsilon$.

Последовательность $0101$ является словом длины 4 над приведенным выше бинарным алфавитом,
а $2 * (3 + 3)$ -- слово длины 7 над алфавитом языка арифметических выражений.
\begin{definition}
$\Sigma^k$ -- множество цепочек длины $k$ над алфавитом $\Sigma$.
\end{definition}
\begin{definition}
$\Sigma^* = \bigcup\limits_{k=0}^\infty$ -- множество всех цепочек над алфавитом $\Sigma$.
\end{definition}
\begin{definition}
Формальный язык над алфавитом $\Sigma$ -- некоторое подмножество $\Sigma^*$.
\end{definition}
В качестве примера формального языка можно привести язык палиндромов -- слов, которые читаются одинаково как слева направо, так и справа налево.
Палиндромами над бинарным алфавитом будут строки $010$ и $1001$.

Заметим, что любой язык программирования является формальным языком, а корректная программа на нем -- словом этого языка.
\subsection{Контекстно-свободные грамматики}
Контекстно-свободная грамматика состоит из следующих компонентов:
\begin{itemize}
    \item алфавит, элементы которого называют терминалами;
    \item конечное множество переменных, называемых нетерминалами; каждый нетерминал представляет формальный язык;
    \item стартовый символ грамматики -- один из нетерминалов, представляющий определямый язык;
    \item конечное множество правил вывода, или продукций, представляющих рекурсивное определение языка.
    Правило вывода представляет из себя пару из нетерминала, называемого головой продукции, и конечной цепочкой,
    состоящей из терминалов и нетерминалов, называемой телом продукции.
\end{itemize}
Формальное определение контекстно-свободной грамматики выглядит следующим образом.
\begin{definition}
Контекстно-свободная грамматика $G$ -- четверка $(\Sigma, N, S \in N, P \subset N \times (\Sigma \cup N)^*)$,
где $\Sigma$ -- алфавит, $N$ - множество нетерминалов, $S$ -- стартовый нетерминал, $P$ -- множество правил вывода.
\end{definition}
Типичным примером контекстно-свободной грамматики является грамматика арифметических выражений, представленная на рисунке~\ref{cf-expr}.
\begin{figure}
    \begin{align*}
        E &\rightarrow E + E\\
        E &\rightarrow T\\
        T &\rightarrow T * T\\
        T &\rightarrow F\\
        F &\rightarrow (E)\\
        F &\rightarrow \pmb{n}
    \end{align*}
    \caption{Конекстно-свободная грамматика языка арифметических выражений}
    \label{cf-expr}
\end{figure}

Для сокращения записи правила с одинаковой головой продукции часто объединяют. На рисунке~\ref{cf-expr-short} представлена сокращенная запись грамматики арифметических выражений.
\begin{figure}
    \begin{align*}
        E &\rightarrow E + E\ |\ T\\
        T &\rightarrow T * T\ |\ F\\
        F &\rightarrow (E)\ |\ \pmb{n}
    \end{align*}
    \caption{Сокращенная запись конекстно-свободной грамматики языка арифметических выражений}
    \label{cf-expr-short}
\end{figure}
\FloatBarrier
\section{Необходимые элементы теории построения компиляторов}
В данном разделе приведено описание компонентов компилятора, необходимых для построения синтаксического анализатора.
Подробнее о построении компиляторов можно узнать в книге~\cite{dragon}.
\subsection{Лексический анализатор}
Лексический анализ или сканирование -- первая фаза компиляции.
В процессе лексического анализа исходная программа, представляющая собой поток символов, разбивается на значащие последовательности, называемые лексемами.
Для каждой лексемы строится выходной токен, представляющий собой пару  <<имя токена-значение атрибута>>.
Имя токена представляет собой абстрактный символ и определяет, к какому классу относится данный токен.
Значение атрибута представляет собой некоторое значение, связанное с данным токеном.

В языке арифметических выражений можно выделить два типа токенов:
\begin{itemize}
    \item операторы -- к данному классу относятся символы <<$+$>>, <<$*$>>, <<$($>> и <<$)$>>, которые являются корректными значениями атрибута для данного класса;
    \item числовые константы -- в данном случае сама числовая константа является значением атрибута.
\end{itemize}
Например, выражение $2 * (3 + 3)$ будет преобразовано в следующую последовательность токенов:
\begin{align*}
&<number, 2>\\&<operator, *>\\&<operator, (>\\&<number, 3>\\&<operator, +>\\&<number, 3>\\&<operator, )>.
\end{align*}
\subsection{Синтаксический анализатор}
Синтаксический анализ или разбор -- вторая фаза компилятора.
В процессе синтаксического на основе первых компонентов токенов, полученных при лексическом анализе, строится древовидное промежуточное представление,
которое описывает грамматическую структуру потока токенов.
Полученная древовидная структура называется деревом разбора, и для ее описания используются контекстно-свободные грамматики.

На рисунке~\ref{parse-tree-full} приведено полное дерево разбора выражения $2 * (3 + 3)$.
Для описания грамматической структуры использовалась контекстно-свободная грамматика, представленная на рисунке~\ref{cf-expr-short}.

Так как дерево разбора задает порядок операций, то обычно в результирующем дереве разбора операторы <<$($>> и <<$)$>> опускаются.
Кроме того, некоторые нетерминалы могут быть заменены на терминалы, соответствующие операции, которая должна быть произведена при раскрытии нетерминала по соответствующиему правилу.
Сокращенная версия дерева разбора выражения $2 * (3 + 3)$ представлена на рисунке~\ref{parse-tree-short}.
% Set the overall layout of the tree
\tikzstyle{level 1}=[level distance=1.5cm, sibling distance=2.5cm]
\tikzstyle{level 2}=[level distance=1.5cm, sibling distance=2.5cm]

% Define styles for bags and leafs
\tikzstyle{bag} = [text width=4em, text centered]
\tikzstyle{end} = [circle, minimum width=3pt,fill, inner sep=0pt]
\begin{figure}
\centering

% The sloped option gives rotated edge labels. Personally
% I find sloped labels a bit difficult to read. Remove the sloped options
% to get horizontal labels. 
\begin{tikzpicture}[grow=down, sloped]
\node[bag] {$E$}
    child {
        node [bag] {$T$}
        child {
            node [bag] {$T$}
            child {
                node [bag] {$F$}
                child {
                    node [end] {}
                    node [below] {$2$}
                }
            }
        }
        child {
            node [end] {}
            node [below] {$*$}
        }
        child {
            node [bag] {$T$}
            child {
                node [bag] {$F$}
                child {
                    node [end] {}
                    node [below] {$($}
                }
                child {
                    node [bag] {$E$}
                    child {
                        node [bag] {$E$}
                        child {
                            node [bag] {$T$}
                            child {
                                node [bag] {$F$}
                                child {
                                    node [end] {}
                                    node [below] {$3$}
                                }
                            }
                        }
                    }
                    child {
                        node [end] {}
                        node [below] {$+$}
                    }
                    child {
                        node [bag] {$E$}
                        child {
                            node [bag] {$T$}
                            child {
                                node [bag] {$F$}
                                child {
                                    node [end] {}
                                    node [below] {$3$}
                                }
                            }
                        }
                    }
                }
                child {
                    node [end] {}
                    node [below] {$)$}
                }
            }
        }
    };
\end{tikzpicture}
\caption{Полное дерево разбора арифметического выражения.}
\label{parse-tree-full}
\end{figure}

\begin{figure}
\centering

% The sloped option gives rotated edge labels. Personally
% I find sloped labels a bit difficult to read. Remove the sloped options
% to get horizontal labels. 
\begin{tikzpicture}[grow=down, sloped]
\node[bag] {$*$}
    child {
        node [end] {}
        node [below] {$2$}
    }
    child {
        node [bag] {$+$}
        child {
            node [end] {}
            node [below] {$3$}
        }
        child {
            node [end] {}
            node [below] {$3$}
        }
    };
\end{tikzpicture}
\caption{Сокращенное дерево разбора арифметического выражения.}
\label{parse-tree-short}
\end{figure}
\section{Featherweight Java}
Featherweigth Java -- чистый язык программирования,
который является минимальным ядром для моделирования системы типов языка Java.
\subsection{Синтаксис Featherweight Java}
На рисунке~\ref{fj-syntax} приведен синтаксис определений классов, конструкторов, методов,
а также синтаксис выражений языка FJ. Здесь и далее будут использоваться следующие обозначения:
\begin{itemize}
    \item $A, B, C, D, E$ -- названия классов;
    \item $f, g$ -- названия полей классов;
    \item $m$ -- названия методов;
    \item $x$ -- переменные;
    \item $d, e$ -- выражения;
    \item $L$ -- определения классов;
    \item $K$ -- определения конструкторов;
    \item $M$ -- определения методов.
\end{itemize}
Каждый метод содержит неявную переменную $this$, и данное имя не может присутствовать в списке параметров.

Надчеркивание используется для обозначения последовательностей, возможно, пустых.
Так, $\overline{f}$ обозначает $f_1, \ldots, f_n$.
Аналогичным образом расшифровываются $\overline{C}, \overline{x}, \overline{e}$ и~т.\,д.
Через $\overline{C}\ \overline{f}$ будет обозначаться последовательность $C_1 f_1, \ldots, C_n, f_n$,
а через $this.\overline{f} = \overline{f}$ будет обозначаться $this.f_1 = f_1, \ldots, this.f_n = f_n$.
\begin{figure}
    \begin{align*}
        L &::= class\ C\ extends\ C\ \{\overline{C}\ \overline{f};\ K\ \overline{M}\}\\
        K &::= C(\overline{C}\ \overline{f})\{\ super(\overline{f});\ this.\overline{f} = \overline{f};\ \}\\
        M &::= C\ m(\overline{C}\ \overline{x}\{\ return\ e;\ \}\\
        e &::= x\ |\ e.f\ |\ e.m(\overline{e})\ |\ new\ C(\overline{e})\ |\ (C)e
    \end{align*}
    \caption{Сокращенное описание синтаксиса языка Featherweight Java}
    \label{fj-syntax}
\end{figure}
\subsection{Правила редукции}
Отношение редукции будем обозначать $e \rightarrow e'$, читается как <<выражение $e$ редуцируется в выражение $e'$ за один шаг>>.
Символом $\rightarrow^*$ будем обозначать рефлексивное и транзитивное замыкание отношения редукции.

Правила редукции выглядят следующим образом:
\begin{enumerate}
    \item $\cfrac{fields(C) = \overline{C}\overline{f}}{(new\ C(\overline{e})).f_i \rightarrow e_i}$;
    \item $\cfrac{mbody(m, C) = \overline{x}.e_0}{(new\ C(\overline{e})).m(\overline{d}) \rightarrow [\overline{d}/\overline{x}, new\ C(\overline{e})/this]e_0}$;
    \item $\cfrac{C <: D}{(D)(new\ C(\overline{e})) \rightarrow new\ C(\overline{e})}$;
    \item $\cfrac{e_0 \rightarrow e_0'}{e_0.f \rightarrow e_0'.f}$;
    \item $\cfrac{e_0 \rightarrow e_0'}{e_0.m(\overline{e}) \rightarrow e_0'.m(\overline{e})}$;
    \item $\cfrac{e_i \rightarrow e_i'}{e_0.m(\ldots, e_i, \ldots) \rightarrow e_0.m(\ldots, e_i, \ldots)}$;
    \item $\cfrac{e_i \rightarrow e_i'}{new\ C(\ldots, e_i, \ldots) \rightarrow new\ C(\ldots, e_i, \ldots)}$;
    \item $\cfrac{e_0 \rightarrow e_0'}{(C)e_0 \rightarrow (C)e_0'}$.
\end{enumerate}
Есть три правила редукции для базовых операций -- доступа к полю класса, вызова метода и приведения типов.
Через $[e/y]e_0$ обозначается результат подстановки выражения $e$ вместо переменной $y$ в выражении $e_0$.
Через $[\overline{d}/\overline{x}]e_0$ обозначается подстановка списка выражений вместо списка переменных.
\section*{Выводы по главе 1}
\addcontentsline{toc}{section}{Выводы по главе 1}
Рефакторинг является неотъемлемой частью процесса разработки программного обеспечения.
Несмотря на то, что процесс проведения рефакторинга подробно описан и интуитивно понятен,
сами рефакторные преобразования по-прежнему должны проводиться вручную, что неизбежно приводит к ошибкам.

После выхода книги Фаулера <<Рефакторинг. Улучшение существующего кода>> стала популярной тема формализации и верификации рефакторингов,
однако существующие работы либо рассматривают вносимые изменения на уровне абстрактных моделей и общей архитектуры программы,
либо требуют в процессе верификации часть работы проводить вручную.

Статический анализ -- широко применяющийся подход проверки программного обеспечения.
В процессе статического анализа исследуется исходный код программы,
поэтому статические анализаторы включают в себя лексические, синтаксические и семантические анализаторы исследуемого языка программирования.

Featherweight Java -- чистый объектно-ориентированный функциональный язык программирования,
обладающий полностью формализованной, простой и понятной семантикой, что позволяет на его примере изучать свойства современных объектно-ориентированных языков.
\chapter{Описание предлагаемого подхода}
\begin{definition}
Строгая вычислимость
\begin{itemize}
    \item $\overline{e}$ вычислимо, если $\forall e \in \overline{e}\ e$ вычислимо
    \item если $e$ вычислимо, то $e.f$ вычислимо;
    \item если $mbody(m, C) = \overline{x}.e_0$ и вычислимы $e$, $\overline{e}$ и $[\overline{e}/\overline{x}, e/this]e_0$, то $e.m(\overline{e})$ вычислимо;
    \item если $\overline{e}$ вычислимо, то $new\ C(\overline{e})$ вычислимо;
    \item если $e$ вычислимо, то $(C) e$ вычислимо.
\end{itemize}
Если выражение содержит вызов рекурсивного метода (с прямой или косвенной рекурсией), то выражение не является строго вычислимым.
\end{definition}
\begin{definition}
Выражения $e_1$ и $e_2$ назовем эквивалентными, если либо оба выражения не являются строго вычислимыми,
либо оба выражения строго вычислимы и существует выражение $e' : e_1 \rightarrow^* e'$ и $e_2 \rightarrow^* e'$. Эквивалентные выражения будем обозначать $e_1 \equiv e_2$.
\end{definition}
\begin{definition}
Пусть в классе $C$ определен метод $m : mbody(m, C) = \overline{x}.e_0$.
Назовем метод $m$ хорошо определенным, если $e_0$ содержит все параметры метода,
включая неявный параметр $this$. Хорошо определенные методы будем обозначать $wd(m, C)$.
\end{definition}
\begin{definition}
Слабая редукция
\begin{enumerate}
    \item $\cfrac{\Gamma \vdash e : C \qquad mbody(m, C) = \overline{x}.e_0 \qquad wd(m, C)}{e.m(\overline{d}) \rightarrow_w [\overline{d}/\overline{x}, e/this]e_0}$
    \item $\cfrac{C <: D}{(D)(new\ C(\overline{e})) \rightarrow_w new\ C(\overline{e})}$
    \item $\cfrac{e \rightarrow_w e'}{e.f \rightarrow_w e'.f}$
    \item $\cfrac{e \rightarrow_w e'}{e.m(\overline{e}) \rightarrow_w e'.m(\overline{e})}$
    \item $\cfrac{e_i \rightarrow_w e_i'}{e.m(\ldots,e_i,\ldots) \rightarrow_w e.m(\ldots,e_i',\ldots)}$
    \item $\cfrac{e_i \rightarrow_w e_i'}{new\ C(\ldots,e_i,\ldots) \rightarrow_w new\ C(\ldots,e_i',\ldots)}$
    \item $\cfrac{e \rightarrow_w e'}{(C)e \rightarrow_w (C)e'}$
\end{enumerate}
\end{definition}

\begin{lemma}
Пусть $e \rightarrow_w e'$. Тогда $e$ строго вычислим тогда и только тогда, когда $e'$ строго вычислим.
\end{lemma}

\begin{proof}
Для первого правила слабой редукции доказательство следует из того, что метод $m$ -- хорошо определен.
Для всех правил слабой редукции, кроме первого, доказательство следует непосредственно из определения строгой вычислимости.
\end{proof}

\begin{theorem}
Пусть $e \rightarrow_w e'$. Тогда $e \equiv e'$.
\end{theorem}

\chapter{Практическая реализация}
\section{Разбор Featherweight Java}
\subsection{Лексический анализатор}
Для разбиения исходного кода на токены использовался генератор лексических анализаторов Alex~\cite{alex}.
\subsection{Синтаксический анализатор}
\begin{figure}[H]
    \begin{align*}
        Classes \rightarrow&\ Class\ Classes\\
        Class \rightarrow&\ class\ ClassName\ extends\ ClassName\\
        &\ \{\ Fields\ Constructor\ Methods\ \}\\
        Fields \rightarrow&\ Field\ Fields\ |\ \varepsilon\\
        Field \rightarrow&\ ClassName\ FieldName;\\
        Constructor \rightarrow& \ ClassName\ (\ Parameters\ )\\
        &\ \{\ super\ (\ Expressions\ );\ Assigns\ \}\\
        Assigns \rightarrow&\ Assign\ Assigns\ |\ \varepsilon\\
        Assing \rightarrow&\ this\ .\ FieldName\ =\ VarName;\\
        Methods \rightarrow&\ Method\ Methods\ |\ \varepsilon\\
        Method \rightarrow&\ ClassName\ MethodName\ (\ Parameters\ )\\
        &\ \{\ return\ Expression;\ \}\\
        Expression \rightarrow&\ VarName\\
        &\ |\ Expression\ .\ FieldName\\
        &\ |\ Expression\ .\ MethodName\ (\ Expressions\ )\\
        &\ |\ new\ ClassName\ (\ Expressions\ )\\
        &\ |\ (\ ClassName\ )\ Expression\\
        Parameters \rightarrow&\ Parameter\ ,\ Parameters\ |\ Parameter\\
        Parameter \rightarrow&\ ClassName\ VarName\\
        Expressions \rightarrow&\ Expression\ ,\ Expressions\ |\ Expression\\
        ClassName \rightarrow&\ id\\
        FieldName \rightarrow&\ id\\
        MethdoName \rightarrow&\ id\\
        VarName \rightarrow&\ id
    \end{align*}
    \caption{Контекстно-свободная грамматика языка Featherweight Java}
    \label{cf-fj}
\end{figure}
Для построения дерева разбора использовался генератор разборщиков Happy~\cite{happy}.
\subsection{Семантический анализатор}

\section{Верификация выделения метода}
\chapter{Примеры практического использования}

\section{Выделение метода}
В листингах~\ref{pf-before} и~\ref{pf-after} приведен код до и после выделения метода $createPair$.
\begin{lstlisting}[float,label=pf-before,caption=Код до выделения метода]
class Pair extends Object {
    Object first;
    Object second;

    Pair(Object first, Object second) {
        super();
        this.first = first;
        this.second = second;
    }
}

class PairFactory extends Object {
    PairFactory() {
        super();
    }

    Pair createPairFst(Object fst) {
        return new Pair(fst, new Object());
    }

    Pair createPairSnd(Object snd) {
        return new Pair(new Object(), snd);
    }
}
\end{lstlisting}

\begin{lstlisting}[float,label=pf-after,caption=Код после выделения метода]
class PairFactory extends Object {
    PairFactory() {
        super();
    }

    Pair createPair(Object fst, Object snd) {
    	return new Pair(fst, snd);
    }

    Pair createPairFst(Object fst) {
        return this.createPair(fst, new Object());
    }

    Pair createPairSnd(Object snd) {
        return this.createPair(new Object(), snd);
    }
}
\end{lstlisting}


Процесс верификации.
\begin{enumerate}
    \item Метод $createPair$ не содержит прямой рекурсии.
    \item В классе $Object$ метод $createPair$ отсутствует.
    \item Дочерние классы отсутствуют.
    \item Вызов выделенного метода присутствует в методах $createPairFst$ и $createPairSnd$.
    \begin{itemize}
        \item Для метода $createPairFst$ выполняется
        $$this.createPair(fst,\ new\ Object()) \rightarrow_m new\ Pair(fst,\ new\ Object()).$$
        В данном случае была произведена подставновка
        $$[fst/fst,\ new\ Object()/snd]new\ Pair(fst,\ snd)$$
        вместо вызова метода $createPair$.
        \item Для метода $createPairSnd$ выполняется
        $$this.createPair(new\ Object(),\ snd) \rightarrow_m new\ Pair(new\ Object(),\ snd).$$
        В данном случае была произведена подставновка
        $$[new\ Object()/fst,\ snd/snd]new\ Pair(fst,\ snd)$$
        вместо вызова метода $createPair$.
    \end{itemize}
\end{enumerate}

\section{Замена условного оператора полиморфизмом}
Введем натуральные числа по аналогии с лямбда-исчислением.
\begin{lstlisting}[float,label=nat,caption=Определение натуральных чисел]
class Nat extends Object {
    Nat() {
       super();
    }

    Bool isZero() {
        return (Bool) new Exception().throw();
    }

    Nat add(Nat value)

    Bool equals(Nat value)
}

class Zero extends Nat {
    Zero() {
        super();
    }

    Bool isZero() {
        return True;
    }

    Nat add(Nat value) {
        return value;
    }

    Bool equals(Nat value) {
        return value.isZero();
    }
}

class Succ extends Nat {
    Nat prev;
    Succ(Nat prev) {
        super();
        this.prev = prev;
    }

    Bool isZero() {
        return False;
    }

    Nat add(Nat value) {
        return prev.add(Succ(value));
    }

    Bool equals(Nat value) {
        return value.isZero().choose(
            new False(),
            prev.equals(((Succ) value).prev));
    }
}
\end{lstlisting}

В качестве иллюстрации приведем классический пример из книги Фаулера <<Рефакторинг. Улучшение существующего кода>>.
В листингах~\ref{et-before} и~\ref{et-after} приведен код до и после замены switch-выражения в методе $payAmount$ полиморфизмом.

\begin{lstlisting}[float,label=et-before,caption=Код до рефакторинга]
class Employee extends Object {
    ...

    Nat getMonthlySalary() { ... }

    Nat getCommission() { ... }

    Nat getBonus() { ... }
}

class EmployeeType {
    ...

    Nat payAmount(Employee emp) {
        return (Nat) new Switch().switch(new SwitchList(new SwitchLeafNode())
            .append(
                this.getTypeCode().equals(0),
                emp.getMonthlySalary()),
            .append(
                this.getTypeCode().equals(1), 
                emp.getMonthlySalary().add(emp.getCommission())),
            .append(
                this.getTypeCode().equals(2),
                emp.getMonthlySalary().add(emp.getBonus())));
    }

    Nat getTypeCode() {
        return (Nat) new Exception().throw();        
    }
}

class Engineer extends EmployeeType {
    ...

    Nat getTypeCode() {
        return 0;
    }
}

class Salesman extends EmployeeType {
    ...

    Nat getTypeCode() {
        return 1;
    }
}

class Manager extends EmployeeType {
    ...

    Nat getTypeCode() {
        return 2;
    }
}
\end{lstlisting}

\begin{lstlisting}[float,label=et-after,caption=Код после рефакторинга]
class EmployeeType {
    ...

    Nat payAmount(Employee emp) {
        return (Nat) new Exception().throw()
    }
}

class Engineer extends EmployeeType {
    ...

    Nat payAmount(Employee emp) {
        return emp.getMonthlySalary();
    }
}

class Salesman extends EmployeeType {
    ...

    Nat payAmount(Employee emp) {
        return emp.getMonthlySalary().add(emp.getCommission());
    }
}

class Manager extends EmployeeType {
    ...

    Nat payAmount(Employee emp) {
        return emp.getMonthlySalary().add(emp.getBonus());
    }
}
\end{lstlisting}
Процесс верификации.

\begin{enumerate}
    \item Тело метода $payAmount$ представляет собой switch-выражение.
    \item Число ветвей в switch-выражении и число дочерних классов совпадают.
    \item Значения условных выражений для всех дочерних классов приведены в таблице~\ref{et-c}.
    \begin{table}[H]
    \begin{center}
    \begin{tabular}{|c|c|c|c|}
    \hline
    Номер ветви & 1 & 2 & 3\\
    \hline
    $Engineer$ & True & False & False\\
    \hline
    $Salesman$ & False & True & False\\
    \hline
    $Manager$ & False & False & True\\
    \hline
    \end{tabular}
    \end{center}
    \caption{Значения условных выражений для дочерних классов класса $EmployeeType$}
    \label{et-c}
    \end{table}
    \item Исходя из результатов предыдущего пункта, классу $Engineer$ ставится в соответствие первая ветвь switch-выражения, классу $Salesman$ -- вторая, $Manager$ -- третья.
    \item Каждая ветвь поставлена в соответствие только одному классу.
    \item Для всех дочерних классов тело метода после рефакторинга совпадает с результатом соответствующей ветви switch-выражения.
\end{enumerate}
\startconclusionpage
В работе предложены алгоритмы верификации двух методов рефакторинга -- выделение метода и замена условного оператора полиморфизмом --
для программ на языке Featherweight Java. Доказана корректность предложенных алгоритмов.

Были показаны примеры использования предложенных алгоритмов как для подтверждения корректности проведенного рефакторинга,
так и для поиска ошибок при его проведении.

\printbibliography

\end{document}
