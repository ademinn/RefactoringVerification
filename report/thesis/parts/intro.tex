\startprefacepage
% \chapter{Введение}

В настоящее время рефакторинг~\cite{refactoring} является неотъемлемой частью процесса разработки программного обеспечения.
Рефакторинг проводят с целью улучшения читаемости кода и изменения его структуры для упрощения дальнейшего расширения.
Для проверки корректности рефакторных преобразований используются модульные тесты~\cite{tdd},
которые необходимо запускать после каждого преобразования.
Данный метод имеет свои недостатки -- в процессе тестирования можно лишь найти ошибки, но не доказать их отсутствие. 
Кроме того, требуется тратить время на написание и поддержку тестов.

Формальная верификация -- формальное доказательство соотетствия или несоответствия формального предмета верификации его формальному описанию.
Формальная верификация, как правило, используется при проверке ПО, к качеству которого предъявляются высокие требования (life-critical system).
Примерами такого ПО являются ПО медицинского оборудования.
Часто для формальной верификации используется статический анализ кода -- анализ, проводимый без реального выполнения программы.
Однако, как правило, статический анализ используется только для проверки самого исходного кода, а не изменений, вносимых в него в процессе разработки.

В настоящей работе предложен метод статической верификации рефакторных преобразований структуры программ.
Верификация проводится для программ на языке программирования Featherweight Java~\cite{fj}.
Для FJ определены правила редукции для всех типов выражений, что позволяет в процессе верификации вносить изменения в программы, не меняя их поведеия.
Так как FJ -- чистый функциональный язык программирования, то результаты, полученные в данной работе,
могут быть применены и к другим чистым функциональным языкам.