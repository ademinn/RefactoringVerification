\chapter{Описание предлагаемого подхода}

По определению, при проведении рефакторинга поведение программы не должно изменяться.
Поэтому, прежде чем описать алгоритмы верификации рефакторингов, требуется формализовать понятие <<одинаковое поведение>>.
Для этого введем понятие эквивалентности по редукции.
\begin{definition}
Выражения $\Gamma \vdash e_1$ и $\Gamma \vdash e_2$ назовем эквивалентными по редукции,
если для любых значений переменных из $\Gamma$ либо существует выражение $e'$ такое, что $e_1 \rightarrow^* e'$ и $e_2 \rightarrow^* e'$,
либо ни $e_1$ ни $e_2$ не могут быть редуцированы до нормальной формы.
Эквивалентные по редукции выражения будем обозначать $e_1 \equiv_r e_2$.
\end{definition}

\section{Выделение метода}

По определению, при выделении метода фрагмент кода, который можно сгруппировать, преобразуется в метод.
Выделенный метод должен находиться в том же классе, где первоначально находился выделенный фрагмент.
Если выделенный метод должен находиться в другом классе, то далее следует воспользоваться
другим методом рефакторинга -- перемещением метода.

При выделении метода и при его верификации следует учесть следующие детали:
\begin{itemize}
    \item тело выделенного метода не может содержать прямой рекурсии -- до проведения рефакторинга выделяемый метод еще не существовал, а значит выделенный фрагмент не мог содержать его вызовов;
    \item в родительских классах не должен быть определен метод с тем же именем -- в противном случае изменится поведение класса, в котором производится выделение метода;
    \item в дочерних классах не должен быть определен метод с тем же именем -- в противном случае в дочерних классах изменится поведение методов, в которых используется выделяемый метод.
\end{itemize}

Для начала определим внутреннюю редукцию по методу,
которая представляет собой подстановку тела метода при его вызове в другом методе,
определенном в том же классе или одном из его потомков.
\begin{definition}
Пусть в классе $C$ определен метод $m$, $mbody(m, C) = \overline{x}.e_m$.
Отношением редукции по методу $m$ назовем отношение со следующими правилами:
\begin{enumerate}
    \item $\cfrac{\Gamma \vdash this : D \qquad D <: C \qquad \overline{e} \rightarrow_m \overline{e}'}{this.m(\overline{e}) \rightarrow_m [\overline{e}'/\overline{x}]e_m}$
    \item $\cfrac{e \rightarrow_m e'}{e.f \rightarrow_m e'.f}$
    \item $\cfrac{e \rightarrow_m e' \qquad \overline{e} \rightarrow_m \overline{e}'}{e.m'(\overline{e}) \rightarrow_m e'.m'(\overline{e}')}$
    \item $\cfrac{\overline{e} \rightarrow_m \overline{e}'}{new\ C(\overline{e}) \rightarrow_m new\ C(\overline{e}')}$
    \item $\cfrac{e \rightarrow_m e'}{(D)e \rightarrow_m (D)e'}$
\end{enumerate}
\end{definition}
Заметим, что подстановка будет происходить, только если редукция проводится для тела одного из методов класса $C$ или его потомков.
В противном случае, для выражения $e$ существует только одно выражение, с которым оно состоит в отношении редукции по методу $m$ -- само выражение $e$.

При помощи редукции по методу введем понятие внутренней подстановки, которое необходимо для доказательства корректности алгоритма верификации выделения метода.
\begin{definition}
Пусть в классе $C$ определены методы $m$ и $m_0$, $mbody(m_0, C) = \overline{x}.e_0$.
Пусть после внесения изменений метод $m_0$ преобразовался в метод $m_0'$, $mbody(m_0', C) = \overline{x}.e_0'$, причем $e_0 \rightarrow_m e_0'$.
Такое преобразование будем называть внутренней подстановкой и обозначать $m_0' = inner_m(m_0)$.
\end{definition}

\begin{theorem}
Пусть в классе $C$ определены методы $m$ и $m_0$ и метод $m$ не содержит прямой рекурсии. Пусть $m_0' = inner_m(m_0)$.
Тогда, $\forall e : C, \overline{e}$ верно: $e.m_0(\overline{e}) \equiv_r e.m_0'(\overline{e})$.
\end{theorem}
\begin{proof}
Пусть $e$ не может быть редуцирован к виду $new\ C(\overline{e}')$. Тогда не могут быть редуцированы вызовы методов $m_0$ и $m_0'$, а значит оба выражения не могут быть редуцированы до нормальной формы.

Пусть $e$ может быть редуцирован к виду $new\ C(\overline{e}')$.

Проведем редукцию вызова метода $m_0$ по второму правилу редукции, затем в полученном выражении проведем редукцию всех вызовов методов $m$ тоже по второму правилу
(это можно сделать, так как на предыдущем шаге была произведена подстановка выражения $new\ C(\overline{e}')$ вместо переменной $this$). Полученное выражение обозначим $e_0$.

Проведем редукцию вызова метода $m_0'$ по второму правилу редукции, обозначим полученное выражение $e_0'$.

Из определения редукции по методу следует, что выражения $e_0$ и $e_0'$ равны.
В самом деле, процесс их вывода отличается лишь порядком проведения подстановок -- для $m_0$ сначала вместо $this$ было подставлено $new\ C(\overline{e}')$, а затем раскрыт вызов метода $m$;
для $m_0'$ сначала был раскрыт вызов метода $m$ (из определения внутренней подстановки), а затем вместо $this$ было подставлено $new\ C(\overline{e}')$.

Значит $e.m_0(\overline{e}) \equiv_r e.m_0'(\overline{e})$.
\end{proof}

Таким образом, алгоритм верификации выделения метода выглядит следующим образом.
\begin{enumerate}
    \item Проверить, что выделенный метод не содержит прямой рекурсии.
    \item Проверить, что родительские классы не содержат метод с таким же именем.
    \item Проверить, что дочерние классы не содержат метод с таким же именем.
    \item Для всех методов, содержащих вызовы выделенного метода, проверить, что тело метода до и после рефакторинга находятся в отношении редукции по методу. Проверка осуществляется конструктивным методом.
\end{enumerate}

\section{Замена условного оператора полиморфизмом}
Согласно определению рассматриваемого метода рефакторинга, происходит преобразование условного оператора, поведение которого зависит от типа объекта (на практике чаще всего используется оператор switch).
Каждая ветвь условного оператора перемещается в перегруженный метод подкласса, а метод базового класса объявляется абстрактным.

\subsection{Ошибки времени выполнения}
Заметим, что в языке Featherweight Java нет понятия <<абстрактный метод>>. Для обозначения того, что метод является абстрактным, вызов такого метода должен приводить к ошибке выполнения.
Такого поведения можно добиться, например, используя приведение типов. В листинге~\ref{exception} приведено определение класса, использующегося для вызова ошибок выполнения.
\begin{lstlisting}[float=htb,label=exception,caption=Определение класса Exception.]
class Exception extends Object {
    Exception() {
        super();
    }

    Exception throw() {
        return (Exception) new Object();
    }
}
\end{lstlisting}

\subsection{Реализация switch-выражения}
В языке Featherweight Java нет примитивного логического типа данных, и соответственно нет условных операторов. Однако, их можно реализовать средствами самого языка, как это делается в лямбда-исчислении.
В листингах~\ref{bool-base},~\ref{bool-true} и~\ref{bool-false} привден пример реализации логического типа данных. Заметим, что метод $choose$ представляет собой условное выражение, и, по сути, заменяет конструкцию $if-then-else$,
использующуюся в других языках.

\begin{lstlisting}[float,label=bool-base,caption=Определение базового класса логического типа данных.]
class Bool extends Object {
    Bool() { super(); }

    Object choose(Object trueExpr, Object falseExpr) {
        return new Exception().throw();
    }

    Bool and(Bool value) {
        return (Bool) new Exception().throw();
    }

    Bool or(Bool value) {
        return (Bool) new Exception().throw();
    }

    Bool not() {
        return (Bool) new Exception().throw();
    }

    Bool equals(Bool value) {
        return this.and(value).or(this.not().and(value.not()));
    }
}
\end{lstlisting}
\begin{lstlisting}[float,label=bool-true,caption=Определение класса True.]
class True extends Bool {
    True() { super(); }

    Object choose(Object trueExpr, Object falseExpr) {
        return trueExpr;
    }

    Bool and(Bool value) {
        return value;
    }

    Bool or(Bool value) {
        return new True();
    }

    Bool not() {
        return new False();
    }
}
\end{lstlisting}
\begin{lstlisting}[float,label=bool-false,caption=Определение класса False.]
class False extends Bool {
    False() { super(); }

    Object choose(Object trueExpr, Object falseExpr) {
        return falseExpr;
    }

    Bool and(Bool value) {
        return new False();
    }

    Bool or(Bool value) {
        return value;
    }

    Bool not() {
        return new True();
    }
}
\end{lstlisting}
Теперь перейдем к определению switch-выражений. Любое switch-выражение представляет собой список пар, состоящих из логического выражения и выражения, представляющего результат.
При вычислении по-очереди проверяются все логические выражения. Результатом будет выражение, соответствующего тому условию, которое оказалось верным и находилось в списке раньше остальных.
Таким образом, для реализации switch-выражений требуется сначала реализовать список пар условие-результат. В листинге~\ref{switch-nodes} приведена реализация элементов такого списка.
\begin{lstlisting}[float,label=switch-nodes,caption=Реализация элементов списка для switch-выражений.]
class SwitchNode extends Object {
    SwitchNode() {
        super();
    }

    Bool isLeaf() {
        return (Bool) new Exception().throw();
    }
}

class SwitchLeafNode extends SwitchNode {
    SwitchLeafNode() {
        super();
    }

    Bool isLeaf() {
        return new True();
    }
}

class SwitchInnerNode extends SwitchNode {
    Bool guard;
    Object value;
    SwitchNode prev;

    SwitchInnerNode(Bool guard, Object value, SwitchNode prev) {
        super();
        this.guard = guard;
        this.value = value;
        this.prev = prev;
    }

    Bool isLeaf() {
        return new False();
    }
}
\end{lstlisting}

В листинге~\ref{switch-list} приведена реализация самого списка, а в листинге~\ref{switch} -- реализация switch-выражений. Так как при приведенной реализации связный список представляет собой стек,
а на практике удобнее использовать очередь, то необходим метод, разворачивающий имеющийся список в обратном порядке.
Данную реализацию можно было бы обобщить, используя структуры данных $List$ и $Pair$, однако,
так как в языке Featherweigth Java отсутствует такая конструкция, как generics~\cite{java},
то это привело бы к большому числу операторов приведения типа.
\begin{lstlisting}[float,label=switch-list,caption=Определение списка пар условие-результат для switch-выражений.]
class SwitchList extends Object {
    SwitchNode top;
    SwitchList(SwitchNode top) {
        super();
        this.top = top;
    }

    SwitchList append(Bool guard, Object value) {
        return new SwitchList(new SwitchInnerNode(guard, value, this.top));
    }

    SwitchList appendAllInnerNode(SwitchInnerNode node) {
        return new SwitchList(new SwitchInnerNode(
            node.guard, node.value, this.top)).appendAll(node.prev);
    }

    SwitchList appendAll(SwitchNode node) {
        return node.isLeaf().choose(
            this,
            this.appendAllInnerNode((SwitchInnerNode) node));
    }

    SwitchList revert() {
        return new SwitchList(new SwitchLeafNode()).appendAll(this.top);
    }

    Bool isEmpty() {
        return this.top.isLeaf();
    }

    SwitchInnerNode getInnerTop() {
        return (SwitchInnerNode) this.top;
    }

    Bool getTopGuard() {
        return this.getInnerTop().guard;
    }

    Object getTopValue() {
        return this.getInnerTop().value;
    }

    SwitchList remove() {
        return new SwitchList(this.getInnerTop().prev);
    }
}
\end{lstlisting}
\begin{lstlisting}[float,label=switch,caption=Определение switch-выражений.]
class Switch extends Object {
    Switch() { super(); }

    Object switchList(SwitchList list) {
        return list.isEmpty().choose(
            new Exception().throw(),
            list.getTopGuard().choose(
                list.getTopValue(),
                this.switchList(list.remove())));
    }

    Object switch(SwitchList list) {
        return this.switchList(list.revert());
    }
}
\end{lstlisting}

\subsection{Алгоритм верификации}
Перейдем к формализации рассматриваемого метода рефакторинга. Прежде чем применять <<замену условного оператора полиморфизмом>>,
следует создать необходимую иерархию наследования. Кроме того, если условное выражение является частью более крупного метода,
то его следует вынести в отдельный метод, используя <<выделение метода>>.

Таким образом, непосредственно перед применением рефакторинга имеется класс $C$, содержащий метод $m$, и классы $D_1, \ldots, D_n$, являющиеся дочерними классами класса $C$.
Тело метода $m$ представляет собой switch-выражение, результат которого зависит от типа объекта.
Должны выполняться два требования:
\begin{itemize}
    \item число ветвей в switch-выражении должно совпадать с числом дочерних классов класса $C$;
    \item все условные выражения в switch-выражении должны приводиться к нормальной форме не более чем за константное число шагов.
\end{itemize}
Исходя из этих требований будем считать, что switch-выражение содержит $n$ пар $(guard_1, result_1), \ldots, (guard_n, result_n)$,
причем для любых $1 \le i, j \le n$ выражение $[new\ D_i(\overline{e})/this]guard_j$ приводится к нормальной форме за константное число шагов.

При проведении верификации прежде всего требуется установить взаимно-однозначное соответствие между ветвями switch-выражения и дочерними классами класса $C$.
Для этого вычисляются значения выражений $[new\ D_i(\overline{e})/this]guard_j$ для всех дочерних классов и для всех условий switch-выражения.
Далее классу $D_i$ ставится в соответствие выражение $guard_j$ такое, что $[new\ D_i(\overline{e})/this]guard_j \rightarrow^* True$,
и для всех $1 \le k < j: [new\ D_i(\overline{e})/this]guard_k \rightarrow^* False$. Если одна и та же ветвь switch-выражения соответствует более чем одному
дочернему классу, то для такого выражения провести рефакторинг невозможно.

Далее остается лишь проверить, что в каждом из дочерних классов после проведения рефакторинга появился метод $m$,
тело которого совпадает с результатом ветви switch-выражения, поставленной в соответствие классу на предыдущем шаге.

Такое преобразование является рефакторингом, так как для любого $i$ верно $new\ D_i(\overline{e_c}).m(\overline{e_m}) \equiv_r new\ D_i'(\overline{e_c}).m(\overline{e_m})$,
где $D_i$ и $D_i'$ -- дочерние классы до и после рефакторинга. В самом деле, все условные выражения приводятся к нормальной форме и результат первой ветви,
для которой условное выражение является истинным, совпадает с телом метода после рефакторинга.

Таким образом, алгоритм верификации замены условного оператора полиморфизмом выглядит следующим образом.
\begin{enumerate}
    \item Проверить, что тело преобразуемого метода представляет собой switch-выражение.
    \item Проверить, что число ветвей в switch-выражении совпадает с числом дочерних классов.
    \item Для каждого дочернего класса вычислить значения условных выражений всех ветвей switch-выражения.
    \item Каждому дочернему классу поставить в соответствие первую ветвь switch-выражения, условное выражение которой является истинным.
    \item Проверить, что каждая ветвь была поставлена в соответствие только одному классу.
    \item Для каждого дочернего класса проверить, что тело метода после рефакторинга совпадает результатом соответствующей ветви switch-выражения.
\end{enumerate}
\section*{Выводы по главе 2}
\addcontentsline{toc}{section}{Выводы по главе 2}
Описаны алгоритмы верификации двух методов рефакторинга -- <<выделение метода>> и <<замена условного оператора полиморфизмом>>.
Доказана корректность предложенных алгоритмов.