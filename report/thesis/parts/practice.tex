\chapter{Практическая реализация}
\section{Разбор Featherweight Java}
\subsection{Лексический анализатор}
Для разбиения исходного кода на токены использовался генератор лексических анализаторов Alex~\cite{alex}.
Разбиение производится на токены следующих типов:
\begin{itemize}
    \item ключевые слова;
    \item операторы;
    \item идентификаторы.
\end{itemize}

К ключевым словам относятся <<$class$>>, <<$return$>>, <<$this$>>, <<$new$>>, <<$extends$>>, <<$super$>>.
Несмотря на то, что в языке Featherweigth Java $this$ рассматривается как неявная переменная,
в процессе синтаксического разбора его удобнее считать ключевым словом, для проверки соответствия конструктора синтаксису.
После синтаксического анализа $this$ считается неявной переменной.

К операторам языка Featherweight Java относятся символы <<$($>>, <<$)$>>, <<$\{$>>, <<$\}$>>, <<$,$>> <<$;$>>, <<$.$>>, <<$=$>>.

К идентификаторам относятся любые последовательности символов, содержащие латинские буквы, цифры, знак подчеркивания, и начинающиеся не с цифры.
Идентификаторами являются, например, названия классов, полей, методов, имена переменных.
\subsection{Синтаксический анализатор}
Перед написанием синтаксического анализатора, необходимо разработать контекстно-свободную грамматику для рассматриваемого языка программирования.
Контекстно-свободная грамматика языка Featherweight Java представлена на рисунке~\ref{cf-fj}.

Для построения синтаксического анализатора использовался генератор разборщиков Happy~\cite{happy}.
Для использования данного разборщика, необходимо избавиться от правого ветвления в грамматике.
Контекстно-свободная грамматика языка Featherweight Java без правого ветвления представлена на рисунке~\ref{cf-fj-happy}.
\begin{figure}[htb]
    \begin{align*}
        Classes \rightarrow&\ Class\ Classes\ |\ \varepsilon\\
        Class \rightarrow&\ class\ ClassName\ extends\ ClassName\\
        &\ \{\ Fields\ Constructor\ Methods\ \}\\
        Fields \rightarrow&\ Field\ Fields\ |\ \varepsilon\\
        Field \rightarrow&\ ClassName\ FieldName;\\
        Constructor \rightarrow& \ ClassName\ (\ Parameters\ )\\
        &\ \{\ super\ (\ Expressions\ );\ Assigns\ \}\\
        Assigns \rightarrow&\ Assign\ Assigns\ |\ \varepsilon\\
        Assign \rightarrow&\ this\ .\ FieldName\ =\ VarName;\\
        Methods \rightarrow&\ Method\ Methods\ |\ \varepsilon\\
        Method \rightarrow&\ ClassName\ MethodName\ (\ Parameters\ )\\
        &\ \{\ return\ Expression;\ \}\\
        Expression \rightarrow&\ VarName\\
        &\ |\ Expression\ .\ FieldName\\
        &\ |\ Expression\ .\ MethodName\ (\ Expressions\ )\\
        &\ |\ new\ ClassName\ (\ Expressions\ )\\
        &\ |\ (\ ClassName\ )\ Expression\\
        Parameters \rightarrow&\ Parameter\ ,\ Parameters\ |\ Parameter\\
        Parameter \rightarrow&\ ClassName\ VarName\\
        Expressions \rightarrow&\ Expression\ ,\ Expressions\ |\ Expression\\
        ClassName \rightarrow&\ id\\
        FieldName \rightarrow&\ id\\
        MethdoName \rightarrow&\ id\\
        VarName \rightarrow&\ id
    \end{align*}
    \caption{Контекстно-свободная грамматика языка Featherweight Java}
    \label{cf-fj}
\end{figure}

\begin{figure}[htb]
    \begin{align*}
        Classes \rightarrow&\ Class\ Classes\ |\ \varepsilon\\
        Class \rightarrow&\ class\ ClassName\ extends\ ClassName\\
        &\ \{\ Members \}\\
        Members \rightarrow&\ Member\ Members\ |\ \varepsilon\\
        Member \rightarrow&\ id\ MemberDefinition\\
        MemberDefinition \rightarrow&\ ConstructorDefinition\\
        &\ |\ id\ Definition\\
        ConstructorDefinition \rightarrow&\ (\ Parameters\ )\\
        &\ \{\ super\ (\ Expressions\ );\ Assigns\ \}\\
        Assigns \rightarrow&\ Assign\ Assigns\ |\ \varepsilon\\
        Assign \rightarrow&\ this\ .\ FieldName\ =\ VarName;\\
        Definition \rightarrow&\ ;\\
        &\ |\ (\ Parameters\ ) \{\ return\ Expression;\ \}\\
        Expression \rightarrow&\ VarName\\
        &\ |\ Expression\ .\ FieldName\\
        &\ |\ Expression\ .\ MethodName\ (\ Expressions\ )\\
        &\ |\ new\ ClassName\ (\ Expressions\ )\\
        &\ |\ (\ ClassName\ )\ Expression\\
        Parameters \rightarrow&\ Parameter\ ,\ Parameters\ |\ Parameter\\
        Parameter \rightarrow&\ ClassName\ VarName\\
        Expressions \rightarrow&\ Expression\ ,\ Expressions\ |\ Expression\\
        ClassName \rightarrow&\ id\\
        FieldName \rightarrow&\ id\\
        MethdoName \rightarrow&\ id\\
        VarName \rightarrow&\ id
    \end{align*}
    \caption{Контекстно-свободная грамматика языка Featherweight Java без правого ветвления}
    \label{cf-fj-happy}
\end{figure}
\subsection{Семантический анализатор}
В процессе семантического анализа производится вывод типов для выражений, проверяется наличие полей и методов.
Также, производится проверка соответствия типов параметров при вызове методов.
В результате семантического анализа дерево разбора преобразуется в абстрактное синтаксическое дерево,
содержащее информацию о типах выражений.
\section{Верификация выделения метода}
Верификация выделения метода проводится согласно алгоритму, описанному во второй главе.