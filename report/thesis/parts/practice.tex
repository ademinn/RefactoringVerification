\chapter{Практическая реализация}
\section{Разбор Featherweight Java}
\subsection{Лексический анализатор}
Для разбиения исходного кода на токены использовался генератор лексических анализаторов Alex~\cite{alex}.
\subsection{Синтаксический анализатор}
\begin{figure}[H]
    \begin{align*}
        Classes \rightarrow&\ Class\ Classes\\
        Class \rightarrow&\ class\ ClassName\ extends\ ClassName\\
        &\ \{\ Fields\ Constructor\ Methods\ \}\\
        Fields \rightarrow&\ Field\ Fields\ |\ \varepsilon\\
        Field \rightarrow&\ ClassName\ FieldName;\\
        Constructor \rightarrow& \ ClassName\ (\ Parameters\ )\\
        &\ \{\ super\ (\ Expressions\ );\ Assigns\ \}\\
        Assigns \rightarrow&\ Assign\ Assigns\ |\ \varepsilon\\
        Assing \rightarrow&\ this\ .\ FieldName\ =\ VarName;\\
        Methods \rightarrow&\ Method\ Methods\ |\ \varepsilon\\
        Method \rightarrow&\ ClassName\ MethodName\ (\ Parameters\ )\\
        &\ \{\ return\ Expression;\ \}\\
        Expression \rightarrow&\ VarName\\
        &\ |\ Expression\ .\ FieldName\\
        &\ |\ Expression\ .\ MethodName\ (\ Expressions\ )\\
        &\ |\ new\ ClassName\ (\ Expressions\ )\\
        &\ |\ (\ ClassName\ )\ Expression\\
        Parameters \rightarrow&\ Parameter\ ,\ Parameters\ |\ Parameter\\
        Parameter \rightarrow&\ ClassName\ VarName\\
        Expressions \rightarrow&\ Expression\ ,\ Expressions\ |\ Expression\\
        ClassName \rightarrow&\ id\\
        FieldName \rightarrow&\ id\\
        MethdoName \rightarrow&\ id\\
        VarName \rightarrow&\ id
    \end{align*}
    \caption{Контекстно-свободная грамматика языка Featherweight Java}
    \label{cf-fj}
\end{figure}
Для построения дерева разбора использовался генератор разборщиков Happy~\cite{happy}.
\subsection{Семантический анализатор}

\section{Верификация выделения метода}