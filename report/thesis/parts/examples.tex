\chapter{Примеры практического использования}

\section{Выделение метода}
\begin{lstlisting}[float=ht,label=pf-before,caption=Код до выделения метода]
class Pair extends Object {
    Object first;
    Object second;

    Pair(Object first, Object second) {
        super();
        this.first = first;
        this.second = second;
    }
}

class PairFactory extends Object {
    PairFactory() {
        super();
    }

    Pair createPairFst(Object fst) {
        return new Pair(fst, new Object());
    }

    Pair createPairSnd(Object snd) {
        return new Pair(new Object(), snd);
    }
}
\end{lstlisting}

\begin{lstlisting}[float=ht,label=pf-after,caption=Код после выделения метода]
class PairFactory extends Object {
    PairFactory() {
        super();
    }

    Pair createPair(Object fst, Object snd) {
    	return new Pair(fst, snd);
    }

    Pair createPairFst(Object fst) {
        return this.createPair(fst, new Object());
    }

    Pair createPairSnd(Object snd) {
        return this.createPair(new Object(), snd);
    }
}
\end{lstlisting}


Процесс верификации.
\begin{enumerate}
    \item Метод $createPair$ не содержит прямой рекурсии.
    \item В классе $Object$ метод $createPair$ отсутствует.
    \item Дочерние классы отсутствуют.
    \item Вызов выделенного метода присутствует в методах $createPairFst$ и $createPairSnd$.
    \begin{itemize}
        \item Для метода $createPairFst$ выполняется $this.createPair(fst,\ new\ Object()) \rightarrow_m new\ Pair(fst,\ new\ Object())$.
        В данном случае была произведена подставновка $[fst/fst,\ new\ Object()/snd]new\ Pair(fst,\ snd)$ вместо вызова метода $createPair$.
        \item Для метода $createPairSnd$ выполняется $this.createPair(new\ Object(),\ snd) \rightarrow_m new\ Pair(new\ Object(),\ snd)$.
        В данном случае была произведена подставновка $[new\ Object()/fst,\ snd/snd]new\ Pair(fst,\ snd)$ вместо вызова метода $createPair$.
    \end{itemize}
\end{enumerate}

\section{Замена условного оператора полиморфизмом}
Введем натуральные числа по аналогии с лямбда-исчислением.
\begin{lstlisting}[float=ht,label=nat,caption=Определение натуральных чисел]
class Nat extends Object {
    Nat() {
       super();
    }

    Bool isZero() {
        return (Bool) new Exception().throw();
    }

    Nat add(Nat value)

    Bool equals(Nat value)
}

class Zero extends Nat {
    Zero() {
        super();
    }

    Bool isZero() {
        return True;
    }

    Nat add(Nat value) {
        return value;
    }

    Bool equals(Nat value) {
        return value.isZero();
    }
}

class Succ extends Nat {
    Nat prev;
    Succ(Nat prev) {
        super();
        this.prev = prev;
    }

    Bool isZero() {
        return False;
    }

    Nat add(Nat value) {
        return prev.add(Succ(value));
    }

    Bool equals(Nat value) {
        return value.isZero().choose(
            new False(),
            prev.equals(((Succ) value).prev));
    }
}
\end{lstlisting}

Классический пример из Фаулера.

\begin{lstlisting}[float=ht,label=pf-before,caption=Код до рефакторинга]
class Employee extends Object {
    ...

    Nat getMonthlySalary() { ... }

    Nat getCommission() { ... }

    Nat getBonus() { ... }
}

class EmployeeType {
    ...

    Nat payAmount(Employee emp) {
        return (Nat) new Switch().switch(new SwitchList(new SwitchLeafNode())
            .append(
                this.getTypeCode().equals(0),
                emp.getMonthlySalary()),
            .append(
                this.getTypeCode().equals(1), 
                emp.getMonthlySalary().add(emp.getCommission())),
            .append(
                this.getTypeCode().equals(2),
                emp.getMonthlySalary().add(emp.getBonus())));
    }

    Nat getTypeCode() {
        return (Nat) new Exception().throw();        
    }
}

class Engineer extends EmployeeType {
    ...

    Nat getTypeCode() {
        return 0;
    }
}

class Salesman extends EmployeeType {
    ...

    Nat getTypeCode() {
        return 1;
    }
}

class Manager extends EmployeeType {
    ...

    Nat getTypeCode() {
        return 2;
    }
}
\end{lstlisting}

\begin{lstlisting}[float=ht,label=pf-before,caption=Код после рефакторинга]
class EmployeeType {
    ...

    Nat payAmount(Employee emp) {
        return (Nat) new Exception().throw()
    }
}

class Engineer extends EmployeeType {
    ...

    Nat payAmount(Employee emp) {
        return emp.getMonthlySalary();
    }
}

class Salesman extends EmployeeType {
    ...

    Nat payAmount(Employee emp) {
        return emp.getMonthlySalary().add(emp.getCommission());
    }
}

class Manager extends EmployeeType {
    ...

    Nat payAmount(Employee emp) {
        return emp.getMonthlySalary().add(emp.getBonus());
    }
}
\end{lstlisting}
Процесс верификации.

\begin{enumerate}
    \item Тело метода $payAmount$ представляет собой switch-выражение.
    \item Число ветвей в switch-выражении и число дочерних классов совпадают.
    \item Значения условных выражений для всех дочерних классов приведены в таблице~\ref{et-c}.
    \begin{table}[H]
    \begin{center}
    \begin{tabular}{|c|c|c|c|}
    \hline
    Номер ветви & 1 & 2 & 3\\
    \hline
    $Engineer$ & True & False & False\\
    \hline
    $Salesman$ & False & True & False\\
    \hline
    $Manager$ & False & False & True\\
    \hline
    \end{tabular}
    \end{center}
    \caption{Значения условных выражений для дочерних классов класса $EmployeeType$}
    \label{et-c}
    \end{table}
    \item Исходя из результатов предыдущего пункта, классу $Engineer$ ставится в соответствие первая ветвь switch-выражения, классу $Salesman$ -- вторая, $Manager$ -- третья.
    \item Каждая ветвь поставлена в соответствие только одному классу.
    \item Для всех дочерних классов тело метода после рефакторинга совпадает с результатом соответствующей ветви switch-выражения.
\end{enumerate}