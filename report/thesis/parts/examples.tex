\chapter{Примеры практического использования}
\section{Выделение метода}
\subsection{Корректно проведенный рефакторинг}
В листингах~\ref{pf-before} и~\ref{pf-after} приведен код до и после выделения метода $createPair$
в методах $createPairFst$ и $createPairSnd$.
\begin{lstlisting}[float,label=pf-before,caption=Код до выделения метода.]
class Pair extends Object {
    Object first;
    Object second;

    Pair(Object first, Object second) {
        super();
        this.first = first;
        this.second = second;
    }
}

class PairFactory extends Object {
    PairFactory() {
        super();
    }

    Pair createPairFst(Object fst) {
        return new Pair(fst, new Object());
    }

    Pair createPairSnd(Object snd) {
        return new Pair(new Object(), snd);
    }
}
\end{lstlisting}
\begin{lstlisting}[float,label=pf-after,caption=Код после выделения метода.]
class PairFactory extends Object {
    PairFactory() {
        super();
    }

    Pair createPair(Object fst, Object snd) {
    	return new Pair(fst, snd);
    }

    Pair createPairFst(Object fst) {
        return this.createPair(fst, new Object());
    }

    Pair createPairSnd(Object snd) {
        return this.createPair(new Object(), snd);
    }
}
\end{lstlisting}

Процесс верификации.
\begin{enumerate}
    \item Метод $createPair$ не содержит прямой рекурсии.
    \item В классе $Object$ метод $createPair$ отсутствует.
    \item Дочерние классы отсутствуют.
    \item Вызов выделенного метода присутствует в методах $createPairFst$ и $createPairSnd$.
    \begin{itemize}
        \item Для метода $createPairFst$ выполняется
        $$this.createPair(fst,\ new\ Object()) \rightarrow_m new\ Pair(fst,\ new\ Object()).$$
        В данном случае была произведена подставновка
        $$[fst/fst,\ new\ Object()/snd]new\ Pair(fst,\ snd)$$
        вместо вызова метода $createPair$.
        \item Для метода $createPairSnd$ выполняется
        $$this.createPair(new\ Object(),\ snd) \rightarrow_m new\ Pair(new\ Object(),\ snd).$$
        В данном случае была произведена подставновка
        $$[new\ Object()/fst,\ snd/snd]new\ Pair(fst,\ snd)$$
        вместо вызова метода $createPair$.
    \end{itemize}
\end{enumerate}
\subsection{Некорректно проведенный рефакторинг}
Предположим, что в предыдущем примере при вынесении метода произошла ошибка, и при создании объекта $Pair$ параметры поменялись местами.
Код неправильно проведенного рефакторинга приведен в листинге~\ref{pf-fail}.

В таком случае, при проведении редукции по методу мы получим
$$this.createPair(fst,\ new\ Object()) \rightarrow_m new\ Pair(new\ Object(),\ fst),$$
$$this.createPair(new\ Object(),\ snd) \rightarrow_m new\ Pair(snd,\ new\ Object()),$$
что не соответствует исходному телу методов $createPairFst$ и $createPairSnd$,
а значит процесс верификации закончится с ошибкой на четвертом шаге.
\begin{lstlisting}[float,label=pf-fail,caption=Код после ошибочного выделения метода.]
class PairFactory extends Object {
    ...

    Pair createPair(Object fst, Object snd) {
        return new Pair(snd, fst);
    }

    ...
}
\end{lstlisting}

\section{Замена условного оператора полиморфизмом}
\subsection{Реализация натуральных чисел}
Согласно определению замены условного оператора полиморфизмом поведение условного оператора должно зависеть от типа объекта.
На практике, как правило, для указания типа объекта используются численные либо строковые константы,
однако в языке Featherweigth Java нет примитивных типов, представляющих числа или печатные символы.

Натуральные числа могут быть введены по аналогии с лямбда-исчислением.
Пример их реализации приведен в листингах~\ref{nat-base},~\ref{nat-zero} и~\ref{nat-succ}.
В таком виде число <<ноль>> будет представлено как $new\ Zero()$, <<один>> как $new\ Succ(new\ Zero())$, <<два>> -- $new\ Succ(new\ Succ(new\ Zero()))$ и~т.~д.
Далее, для простоты, будем использовать стандартные обозначения для числовых констант.
\begin{lstlisting}[float,label=nat-base,caption=Определение базового класса натуральных чисел.]
class Nat extends Object {
    Nat() {
       super();
    }

    Bool isZero() {
        return (Bool) new Exception().throw();
    }

    Nat add(Nat value) {
        return (Nat) new Exception().throw();
    }

    Bool equals(Nat value) {
        return (Bool) new Exception().throw();
    }
}
\end{lstlisting}
\begin{lstlisting}[float,label=nat-zero,caption=Определение класса{,} представляющего число ноль.]
class Zero extends Nat {
    Zero() {
        super();
    }

    Bool isZero() {
        return True;
    }

    Nat add(Nat value) {
        return value;
    }

    Bool equals(Nat value) {
        return value.isZero();
    }
}
\end{lstlisting}
\begin{lstlisting}[float,label=nat-succ,caption=Определение класса{,} представляющего натуральное число{,} следующее за указанным.]
class Succ extends Nat {
    Nat prev;
    Succ(Nat prev) {
        super();
        this.prev = prev;
    }

    Bool isZero() {
        return False;
    }

    Nat add(Nat value) {
        return prev.add(Succ(value));
    }

    Bool equals(Nat value) {
        return value.isZero().choose(
            new False(),
            prev.equals(((Succ) value).prev));
    }
}
\end{lstlisting}
\subsection{Корректно проведенный рефакторинг}
В качестве иллюстрации приведем классический пример из книги Фаулера <<Рефакторинг. Улучшение существующего кода>>.
В листингах~\ref{et-before-e-et} и~\ref{et-before-esm} приведен код до, а в листинге~\ref{et-after} после замены switch-выражения в методе $payAmount$ полиморфизмом.

\begin{lstlisting}[float,label=et-before-e-et,caption=Код до рефакторинга. Определение классов $Employee$ и $EmployeeType$.]
class Employee extends Object {
    ...

    Nat getMonthlySalary() { ... }

    Nat getCommission() { ... }

    Nat getBonus() { ... }
}

class EmployeeType {
    ...

    Nat payAmount(Employee emp) {
        return (Nat) new Switch().switch(new SwitchList(new SwitchLeafNode())
            .append(
                this.getTypeCode().equals(0),
                emp.getMonthlySalary()),
            .append(
                this.getTypeCode().equals(1), 
                emp.getMonthlySalary().add(emp.getCommission())),
            .append(
                this.getTypeCode().equals(2),
                emp.getMonthlySalary().add(emp.getBonus())));
    }

    Nat getTypeCode() {
        return (Nat) new Exception().throw();        
    }
}
\end{lstlisting}
\begin{lstlisting}[float,label=et-before-esm,caption=Код до рефакторинга. Определение наследников класса $EmployeeType$.]
class Engineer extends EmployeeType {
    ...

    Nat getTypeCode() {
        return 0;
    }
}

class Salesman extends EmployeeType {
    ...

    Nat getTypeCode() {
        return 1;
    }
}

class Manager extends EmployeeType {
    ...

    Nat getTypeCode() {
        return 2;
    }
}
\end{lstlisting}

\begin{lstlisting}[float,label=et-after,caption=Код после рефакторинга.]
class EmployeeType {
    ...

    Nat payAmount(Employee emp) {
        return (Nat) new Exception().throw()
    }
}

class Engineer extends EmployeeType {
    ...

    Nat payAmount(Employee emp) {
        return emp.getMonthlySalary();
    }
}

class Salesman extends EmployeeType {
    ...

    Nat payAmount(Employee emp) {
        return emp.getMonthlySalary().add(emp.getCommission());
    }
}

class Manager extends EmployeeType {
    ...

    Nat payAmount(Employee emp) {
        return emp.getMonthlySalary().add(emp.getBonus());
    }
}
\end{lstlisting}
Процесс верификации.

\begin{enumerate}
    \item Тело метода $payAmount$ представляет собой switch-выражение.
    \item Число ветвей в switch-выражении и число дочерних классов совпадают.
    \item Значения условных выражений для всех дочерних классов приведены в таблице~\ref{et-c}.
    \begin{table}[H]
    \begin{center}
    \begin{tabular}{|c|c|c|c|}
    \hline
    Номер ветви & 1 & 2 & 3\\
    \hline
    $Engineer$ & True & False & False\\
    \hline
    $Salesman$ & False & True & False\\
    \hline
    $Manager$ & False & False & True\\
    \hline
    \end{tabular}
    \end{center}
    \caption{Значения условных выражений для дочерних классов класса $EmployeeType$.}
    \label{et-c}
    \end{table}
    \item Исходя из результатов предыдущего пункта, классу $Engineer$ ставится в соответствие первая ветвь switch-выражения, классу $Salesman$ -- вторая, $Manager$ -- третья.
    \item Каждая ветвь поставлена в соответствие только одному классу.
    \item Для всех дочерних классов тело метода после рефакторинга совпадает с результатом соответствующей ветви switch-выражения.
\end{enumerate}
\subsection{Некорректно проведенный рефакторинг}
Предположим, что в предыдущем примере, при перемещении результата ветви switch-выражения, соответствующей классу $Manager$, произошла ошибка,
и метод $payAmount$ для класса $Manager$ стал возврашать то же, что и для класса $Salesman$.
Код неправильно проведенного рефакторинга приведен в листинге~\ref{et-fail}.
Тогда процесс верификации завершится с ошибкой на шестом шаге при сравнении тела метода $payAmount$ класса $Manager$ с
результатом соответствующей ему ветви switch-выражения.
\begin{lstlisting}[float,label=et-fail,caption=Код после некорректно проведенного рефакторинга.]
class Manager extends EmployeeType {
    ...

    Nat payAmount(Employee emp) {
        return emp.getMonthlySalary().add(emp.getCommission());
    }
}
\end{lstlisting}
\section*{Выводы по главе 4}
\addcontentsline{toc}{section}{Выводы по главе 4}
Рассмотрены примеры корректно и некорректно проведенных рефакторингов -- выделения метода и замены условного оператора полиморфизмом.
Показано, что разработанные алгоритмы верификации подтверждают корректность правильно проведенных рефакторингов и заканчиваются с ошибкой
при неправильно проведенных рефакторингах.