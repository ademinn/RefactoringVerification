\chapter{Обзор предметной области}
\section{Необходимые элементы теории формальных языков}
\subsection{Формальные языки}
\begin{definition}
Алфавит -- конечное непустое множество символов.
\end{definition}
Далее для обозначения алфавита будет использоваться символ $\Sigma$.
\begin{definition}
Слово или цепочка -- конечная последовательность символов некоторого алфавита.
\end{definition}
\begin{definition}
Длина цепочки -- число символов в цепочке.
\end{definition}
\begin{definition}
$\Sigma^k$ -- множество цепочек длины $k$ над алфавитом $\Sigma$.
\end{definition}
\begin{definition}
$\Sigma^* = \bigcup\limits_{k=0}^\infty$ -- множество всех цепочек над алфавитом $\Sigma$.
\end{definition}
\begin{definition}
Формальный язык над алфавитом $\Sigma$ -- некоторое подмножество $\Sigma^*$.
\end{definition}
Любой язык прогроммирования является формальным языком, а корректная программа на нем -- словом этого языка.
\subsection{Контекстно-свободные грамматики}
Контекстно-свободная грамматика состоит из следующих компонентов:
\begin{itemize}
    \item алфавит, элементы которого называют терминалами;
    \item конечно множество переменных, называемых нетерминалами; каждый нетерминал представляет формальный язык;
    \item стартовый символ грамматики -- один из нетерминалов, представляющий определямый язык;
    \item конечное множество правил вывода, или продукций, представляющих рекурсивное определение языка.
    Привило вывода представляет из себя пару из нетерминала, называемого головой продукции, и конечной цепочкой,
    состоящей из терминалов и нетерминалов, называемой телом продукции.
\end{itemize}
Формальное определение контекстно-свободной грамматики выглядит следующим образом.
\begin{definition}
Контекстно-свободная грамматика $G$ -- четверка $(\Sigma, N, S \in N, P \subset N \times (\Sigma \cup N)^*)$,
где $\Sigma$ -- алфавит, $N$ - множество нетерминалов, $S$ -- стартовый нетерминал, $P$ -- множество правил вывода.
\end{definition}
Типичным примером контекстно-свободной грамматики является грамматика арифметических выражений, представленная на рисунке~\ref{cf-expr}.
\begin{figure}[H]
    % {\centering
    \centering
    \begin{varwidth}{\textwidth}
    \begin{itemize}
        \item[] $E \rightarrow E + E$
        \item[] $E \rightarrow T$
        \item[] $T \rightarrow T * T$
        \item[] $T \rightarrow F$
        \item[] $F \rightarrow n$
        \item[] $F \rightarrow (E)$
    \end{itemize}
    \end{varwidth}
    % }
    \caption{Конекстно-свободная грамматика языка арифметических выражений}
    \label{cf-expr}
\end{figure}
\section{Редукция}

\section{Featherweight Java}
Featherweigth Java -- чистый функциональный язык программирования,
который является минимальным ядром для моделирования системы типов языка Java.
\begin{figure}[H]
\begin{itemize}
    \item[] $Classes \rightarrow Class\ Classes$
    \item[] $Class \rightarrow class\ ClassName\ extends\ ClassName\ \{\ Fields\ Constructor\ Methods\ \}$
    \item[] $Fields \rightarrow Field\ Fields\ |\ \varepsilon$
    \item[] $Field \rightarrow ClassName\ FieldName;$
    \item[] $Constructor \rightarrow ClassName\ (\ Parameters\ )\ \{\ super\ (\ Expressions\ );\ Assigns\ \}$
    \item[] $Assigns \rightarrow Assign\ Assigns\ |\ \varepsilon$
    \item[] $Assing \rightarrow this\ .\ FieldName\ =\ VarName;$    
    \item[] $Methods \rightarrow Method\ Methods\ |\ \varepsilon$
    \item[] $Method \rightarrow ClassName\ MethodName\ (\ Parameters\ )\ \{\ return\ Expression;\ \}$
    \item[] $Expression \rightarrow VarName\ |\ Expression\ .\ FieldName\ |\ Expression\ .\ MethodName\ (\ Expressions\ )$
    \item[] $Expression \rightarrow new\ ClassName\ (\ Expressions\ )\ |\ (\ ClassName\ )\ Expression$
    \item[] $Parameters \rightarrow Parameter\ ,\ Parameters\ |\ Parameter$
    \item[] $Parameter \rightarrow ClassName\ VarName$
    \item[] $Expressions \rightarrow Expression\ ,\ Expressions\ |\ Expression$
    \item[] $ClassName \rightarrow id$
    \item[] $FieldName \rightarrow id$
    \item[] $MethdoName \rightarrow id$
    \item[] $VarName \rightarrow id$
    \caption{Контекстно-свободная грамматика языка Featherweight Java}
    \label{cf-fj}
\end{itemize}
\end{figure}
\begin{figure}[H]
\begin{itemize}
    \item[] $P = class\ C\ extends\ C\ \{\overline{C}\ \overline{f};\ K\ \overline{M}\}$
    \item[] $K = C(\overline{C}\ \overline{f})\{super(\overline{f});\ this.\overline{f} = \overline{f};\}$
    \item[] $M = C m(\overline{C}\ \overline{x}\{ return e; \}$
    \item[] $e = x\ |\ e.f\ |\ e.m(\overline{e})\ |\ new\ C(\overline{e})\ |\ (C)e$
    \caption{Сокращенное описание синтаксиса языка Featherweight Java}
    \label{fj-syntax}
\end{itemize}
\end{figure}
\section{Рефакторинг}
У термина <<рефакторинг>> существует два определения -- рефакторинг как изменение в коде и рефакторинг как процесс внесения таких изменений.
\begin{definition}
Рефакторинг или рефакторное преобразование -- изменение во внутренней структуре программного обеспечения,
имеющее целью облегчить понимание его работы и упростить модификацию, не затрагивая наблюдаемого поведения.
\end{definition}
\begin{definition}
Производить рефакторинг -- изменять структуру программного обеспечения, применяя
ряд рефакторных преобразований, не затрагивая его поведения.
\end{definition}
Можно выделить несколько основных целей, с которыми проводится рефакторинг:
\begin{itemize}
    \item улучшение композиции ПО -- после проведения рефакторинга улучшается структура кода,
    становится проще вносить изменения в существующий код и добавлять новую функциональность;
    \item облегчение понимания ПО -- в процессе модификации незнакомого кода проще разобраться в деталях реализации;
    \item поиск ошибок -- при рефакторинге требуется глубоко вникать в модифицируемый код, что позволяет обнаружить ошибки;
    кроме того, после прояснения структуры программы некоторые ошибки становятся очевидными;
    \item ускорение разработки -- рефакторинг помогает сохранять хороший дизайн разрабатываемого ПО,
    что положительно сказывается на скорости разработки.
\end{itemize}
\subsection{Методы рефакторинга}
Рефакторное преобразование в общем смысле -- любое преобразование кода, не меняющее его поведения.
Однако существуют шаблонные преобразования, которые встречаются очень часто.
Такие шаблонные преобразования называют методами рефакторинга.
Ниже приводится несколько примеров наиболее часто встречающихся методов рефакторинга.
\subsubsection{Выделение метода}
Выделение метода -- один из наиболее часто проводимых методов рефакторинга.
При выделении метода фрагмент кода, который можно сгруппировать, преобразуется в метод,
название которого обозначает его назначение.

Выделение метода служит сразу двум целям:
\begin{itemize}
    \item избавление от дублирования кода -- при вынесении метода часто выносится код, который используется сразу в нескольких местах;
    кроме того, если выделен мелкий метод, повышается вероятность его использования другими методами;
    \item улучшение читаемости -- название метода служит цели документирования кода,
    и более длинные методы начинают выглядеть, как ряд комментариев.
\end{itemize}
В листингах~\ref{em-before} и~\ref{em-after} представлен пример кода до и после применения метода рефакторинга <<выделение метода>> --
создание копии объекта класса $Pair$ вынесено в метод $copy$.
\lstset{
    language=Java,
    basicstyle=\small\ttfamily,
    frame=single,
    captionpos=b
}
%\begin{minipage}{\linewidth}
\begin{lstlisting}[float,floatplacement=H,label=em-before,caption=Код до проведения рефакторинга]
class A extends Object {
    A() {
        super();
    }
}

class Pair extends Object {
    Object first;
    Object second;

    Pair(Object first, Object second) {
        super();
        this.first = first;
        this.second = second;
    }
}

class PairFactory extends Object {
    Pair template;

    PairFactory(Pair template) {
        super();
        this.template = template;
    }

    Pair createPair() {
        return new Pair(this.template.first, this.template.second);
    }
}
\end{lstlisting}
%\end{minipage}
\begin{lstlisting}[float,floatplacement=h,label=em-after,caption=Код после вынесения метода]
class A extends Object {
    A() {
        super();
    }
}

class Pair extends Object {
    Object first;
    Object second;

    Pair(Object first, Object second) {
        super();
        this.first = first;
        this.second = second;
    }

    Pair copy() {
        return new Pair(this.first, this.second);
    }
}

class PairFactory extends Object {
    Pair template;

    PairFactory(Pair template) {
        super();
        this.template = template;
    }

    Pair createPair() {
        return this.template.copy();
    }
}
\end{lstlisting}
\section{Статическая верификация}
