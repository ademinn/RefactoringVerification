\chapter{Обзор предметной области}
\section{Существующие работы по теме}
Наиболее полное описание методов рефакторинга, способов их применения, цели проведения рефакторинга и его влияния на процесс разработки
приведено в книге Фаулера <<Рефакторинг. Улучшение существующего кода>>.
После ее выхода было предложено несколько способов формализации и верификации рефакторингов,
основанных на проверке инвариантов при внесении изменений~\cite{Rustan04objectinvariants, DBLP:journals/eceasst/MassoniGB06}.
Однако для их применения требуется в явном виде описывать инварианты при написании программы,
что не позволяет полностью автоматизировать процесс верификации.

Рефакторные преобразования архитектуры программ были формализованы для языков описания архитектуры,
таких как Wright~\cite{Allen98specifyingand} и Darwin~\cite{Ehrig:2006:FAG:1121741}.
В работе~\cite{Bisztray_verificationof} рассмотрена верификация архитектурных рефакторингов для моделей,
описанных на языке UML.

В работе~\cite{Taentzer:1998:DCM:645872.668868} вводится понятие <<distributed graph transformation>>,
с помощью которого архитектурные изменения в распределенных системах рассматриваются как
измененения на локальном уровне и на уровне системы в целом.

Еще один способ формализации архитектурных рефакторингов представлен в работе~\cite{Hirsch:1998:GGC:288408.288426}.
В ней система в целом представляется в виде гиперграфа, а вносимые изменения
представляют собой синхронную замену ребер.

Однако все эти подходы могут применяться только при перепроектировании общей архитектуры системы,
представленной на некотором языке высокого уровня, которая лишь описывает общую модель.
Поэтому, данные подходы неприменимы при написании непосредственно исходного кода разрабатываемого программного обеспечения.

\section{Статический анализ кода}
Статический анализ кода -- анализ программного обеспечения, проводимый без рельного выполнения исследуемых программ.
Статические анализаторы существуют для многих популярных языков программирования.
Примерами могут служить PVS-Studio для языка C++ или FindBugs для языка Java.

Статические анализаторы широко применяются в тех областях, где предъявляются высокие требования к качеству программного обеспечения.
Например, статический анализ используется для разработки авиационного ПО~\cite{static-air} или ПО для медицинского оборудования~\cite{static-med}.


Как правило, анализ проводится над какой-либо версией исходного кода,
и статическому анализатору прежде всего требуется построить и проверить корректность абстрактного синтаксического дерева.
Для этого статичесий анализатор либо содержит в себе лесический, синтаксический и семантический анализаторы,
либо, если это возможно, использует один из существующих компиляторов языка, на котором написана исследуемая программа.

Как правило, статический анализ используется для проверки некоторой фиксированной версии исходного кода.
В данной работе предлагается проверять не сам исходный код, а изменения, вносимые в него,
а именно -- рефакторные преобразования, то есть изменения, не затрагивающие поведение программы.
\section{Необходимые элементы теории формальных языков}
В данном разделе приведены определения, необходимые для описания синтаксического анализатора -- одного из компонентов статического анализатора.
Подробнее о теории формальных языков можно узнать в книге~\cite{hmu}.
\subsection{Формальные языки}
\begin{definition}
Алфавит -- конечное непустое множество символов.
\end{definition}
Далее для обозначения алфавита будет использоваться символ $\Sigma$.

Примером простейшего алфавита является бинарный алфавит $\Sigma = \{0, 1\}$.
Еще одним примером является алфавит, состоящий из цифр, круглых скобок и знаков $+, *$ -- алфавит языка арифметических выражений.
\begin{definition}
Слово или цепочка -- конечная последовательность символов некоторого алфавита.
\end{definition}
\begin{definition}
Длина цепочки -- число символов в цепочке.
\end{definition}
Для обозначения цепочки нулевой длины принято использовать греческую букву $\varepsilon$.

Последовательность $0101$ является словом длины 4 над приведенным выше бинарным алфавитом,
а $2 * (3 + 3)$ -- слово длины 7 над алфавитом языка арифметических выражений.
\begin{definition}
$\Sigma^k$ -- множество цепочек длины $k$ над алфавитом $\Sigma$.
\end{definition}
\begin{definition}
$\Sigma^* = \bigcup\limits_{k=0}^\infty$ -- множество всех цепочек над алфавитом $\Sigma$.
\end{definition}
\begin{definition}
Формальный язык над алфавитом $\Sigma$ -- некоторое подмножество $\Sigma^*$.
\end{definition}
В качестве примера формального языка можно привести язык палиндромов -- слов, которые читаются одинаково как слева направо, так и справа налево.
Палиндромами над бинарным алфавитом будут строки $010$ и $1001$.

Заметим, что любой язык программирования является формальным языком, а корректная программа на нем -- словом этого языка.
\subsection{Контекстно-свободные грамматики}
Контекстно-свободная грамматика состоит из следующих компонентов:
\begin{itemize}
    \item алфавит, элементы которого называют терминалами;
    \item конечное множество переменных, называемых нетерминалами; каждый нетерминал представляет формальный язык;
    \item стартовый символ грамматики -- один из нетерминалов, представляющий определямый язык;
    \item конечное множество правил вывода, или продукций, представляющих рекурсивное определение языка.
    Правило вывода представляет из себя пару из нетерминала, называемого головой продукции, и конечной цепочкой,
    состоящей из терминалов и нетерминалов, называемой телом продукции.
\end{itemize}
Формальное определение контекстно-свободной грамматики выглядит следующим образом.
\begin{definition}
Контекстно-свободная грамматика $G$ -- четверка $(\Sigma, N, S \in N, P \subset N \times (\Sigma \cup N)^*)$,
где $\Sigma$ -- алфавит, $N$ - множество нетерминалов, $S$ -- стартовый нетерминал, $P$ -- множество правил вывода.
\end{definition}
Типичным примером контекстно-свободной грамматики является грамматика арифметических выражений, представленная на рисунке~\ref{cf-expr}.
\begin{figure}[htb]
    \begin{align*}
        E &\rightarrow E + E\\
        E &\rightarrow T\\
        T &\rightarrow T * T\\
        T &\rightarrow F\\
        F &\rightarrow (E)\\
        F &\rightarrow n
    \end{align*}
    \caption{Конекстно-свободная грамматика языка арифметических выражений}
    \label{cf-expr}
\end{figure}
Для сокращения записи правила с одинаковой головой продукции часто объединяют. На рисунке~\ref{cf-expr-small} представлена сокращенная запись грамматики арифметических выражений.
\begin{figure}[htb]
    \begin{align*}
        E &\rightarrow E + E\ |\ T\\
        T &\rightarrow T * T\ |\ F\\
        F &\rightarrow (E)\ |\ n
    \end{align*}
    \caption{Сокращенная запись конекстно-свободной грамматики языка арифметических выражений}
    \label{cf-expr-small}
\end{figure}
\section{Необходимые элементы теории построения компиляторов}
В данном разделе приведено описание компонентов компилятора, необходимых для построения синтаксического анализатора.
Подробнее о построении компиляторов можно узнать в книге~\cite{dragon}.
\subsection{Лексический анализатор}
\subsection{Синтаксический анализатор}
\section{Редукция}

\section{Featherweight Java}
Featherweigth Java -- чистый язык программирования,
который является минимальным ядром для моделирования системы типов языка Java.
\subsection{Синтаксис Featherweight Java}
На рисунке~\ref{fj-syntax} приведен синтаксис определений классов, конструкторов, методов,
а также синтаксис выражений языка FJ. Здесь и далее будут использоваться следующие обозначения:
\begin{itemize}
    \item $A, B, C, D, E$ -- названия классов;
    \item $f, g$ -- названия полей классов;
    \item $m$ -- названия методов;
    \item $x$ -- переменные;
    \item $d, e$ -- выражения;
    \item $L$ -- определения классов;
    \item $K$ -- определения конструкторов;
    \item $M$ -- определения методов.
\end{itemize}
Каждый метод содержит неявную переменную $this$, и данное имя не может присутствовать в списке параметров.

Надчеркивание используется для обозначения последовательностей, возможно, пустых.
Так, $\overline{f}$ обозначает $f_1, \ldots, f_n$.
Аналогичным образом расшифровываются $\overline{C}, \overline{x}, \overline{e}$ и~т.\,д.
Через $\overline{C}\ \overline{f}$ будет обозначаться последовательность $C_1 f_1, \ldots, C_n, f_n$,
а через $this.\overline{f} = \overline{f}$ будет обозначаться $this.f_1 = f_1, \ldots, this.f_n = f_n$.
\begin{figure}[htb]
    \begin{align*}
        L &::= class\ C\ extends\ C\ \{\overline{C}\ \overline{f};\ K\ \overline{M}\}\\
        K &::= C(\overline{C}\ \overline{f})\{\ super(\overline{f});\ this.\overline{f} = \overline{f};\ \}\\
        M &::= C\ m(\overline{C}\ \overline{x}\{\ return\ e;\ \}\\
        e &::= x\ |\ e.f\ |\ e.m(\overline{e})\ |\ new\ C(\overline{e})\ |\ (C)e
    \end{align*}
    \caption{Сокращенное описание синтаксиса языка Featherweight Java}
    \label{fj-syntax}
\end{figure}
\subsection{Правила редукции}
Отношение редукции будем обозначать $e \rightarrow e'$, читается как <<выражение $e$ редуцируется в выражение $e'$ за один шаг>>.
Символом $\rightarrow^*$ будем обозначать рефлексивное и транзитивное замыкание отношения редукции.

Правила редукции выглядят следующим образом:
\begin{enumerate}
    \item $\cfrac{fields(C) = \overline{C}\overline{f}}{(new\ C(\overline{e})).f_i \rightarrow e_i}$;
    \item $\cfrac{mbody(m, C) = \overline{x}.e_0}{(new\ C(\overline{e})).m(\overline{d}) \rightarrow [\overline{d}/\overline{x}, new\ C(\overline{e})/this]e_0}$;
    \item $\cfrac{C <: D}{(D)(new\ C(\overline{e})) \rightarrow new\ C(\overline{e})}$;
    \item $\cfrac{e_0 \rightarrow e_0'}{e_0.f \rightarrow e_0'.f}$;
    \item $\cfrac{e_0 \rightarrow e_0'}{e_0.m(\overline{e}) \rightarrow e_0'.m(\overline{e})}$;
    \item $\cfrac{e_i \rightarrow e_i'}{e_0.m(\ldots, e_i, \ldots) \rightarrow e_0.m(\ldots, e_i, \ldots)}$;
    \item $\cfrac{e_i \rightarrow e_i'}{new\ C(\ldots, e_i, \ldots) \rightarrow new\ C(\ldots, e_i, \ldots)}$;
    \item $\cfrac{e_0 \rightarrow e_0'}{(C)e_0 \rightarrow (C)e_0'}$.
\end{enumerate}
Есть три правила редукции для базовых операций -- доступа к полю класса, вызова метода и приведения типов.
Через $[e/y]e_0$ обозначается результат подстановки выражения $e$ вместо переменной $y$ в выражении $e_0$.
Через $[\overline{d}/\overline{x}]e_0$ обозначается подстановка списка выражений вместо списка переменных.
\section{Рефакторинг}
У термина <<рефакторинг>> существует два определения -- рефакторинг как изменение в коде и рефакторинг как процесс внесения таких изменений.
\begin{definition}
Рефакторинг или рефакторное преобразование -- изменение во внутренней структуре программного обеспечения,
имеющее целью облегчить понимание его работы и упростить модификацию, не затрагивая наблюдаемого поведения.
\end{definition}
\begin{definition}
Производить рефакторинг -- изменять структуру программного обеспечения, применяя
ряд рефакторных преобразований, не затрагивая его поведения.
\end{definition}
Можно выделить несколько основных целей, с которыми проводится рефакторинг:
\begin{itemize}
    \item улучшение композиции ПО -- после проведения рефакторинга улучшается структура кода,
    становится проще вносить изменения в существующий код и добавлять новую функциональность;
    \item облегчение понимания ПО -- в процессе модификации незнакомого кода проще разобраться в деталях реализации;
    \item поиск ошибок -- при рефакторинге требуется глубоко вникать в модифицируемый код, что позволяет обнаружить ошибки;
    кроме того, после прояснения структуры программы некоторые ошибки становятся очевидными;
    \item ускорение разработки -- рефакторинг помогает сохранять хороший дизайн разрабатываемого ПО,
    что положительно сказывается на скорости разработки.
\end{itemize}
\subsection{Методы рефакторинга}
Рефакторное преобразование в общем смысле -- любое преобразование кода, не меняющее его поведения.
Однако существуют шаблонные преобразования, которые встречаются очень часто.
Такие шаблонные преобразования называют методами рефакторинга.
Ниже приводится несколько примеров наиболее часто встречающихся методов рефакторинга.
\subsubsection{Выделение метода}
Выделение метода -- один из наиболее часто проводимых методов рефакторинга.
При выделении метода фрагмент кода, который можно сгруппировать, преобразуется в метод,
название которого обозначает его назначение.

Выделение метода служит сразу двум целям:
\begin{itemize}
    \item избавление от дублирования кода -- при вынесении метода часто выносится код, который используется сразу в нескольких местах;
    кроме того, если выделен мелкий метод, повышается вероятность его использования другими методами;
    \item улучшение читаемости -- название метода служит цели документирования кода,
    и более длинные методы начинают выглядеть, как ряд комментариев.
\end{itemize}
\lstset{
    language=Java,
    basicstyle=\small\ttfamily,
    frame=single,
    captionpos=b
}
\section*{Выводы по главе 1}
\addcontentsline{toc}{section}{Выводы по главе 1}
