\documentclass[hyperref=unicode,graphics=pdflatex,13pt]{beamer}

\mode<presentation>
{
%   \usetheme{Goettingen}
%   \setbeamercovered{invisible}
  \useoutertheme{infolines}
%   \setbeamercovered{invisible}
}

\graphicspath{{pic/}}

\usepackage[utf8]{inputenc}
\usepackage[T2A]{fontenc}
\usepackage[english,russian]{babel}
\usepackage{times}
\usepackage{color, colortbl}
\usepackage{tikz}
\usepackage{calc}
\usepackage{array}
\usepackage{multirow}
\usepackage{listings}
\usetikzlibrary{calc,intersections,through,backgrounds,chains,positioning,fit}

%\usepackage{xcolor}

\newcommand\highlight[1]{{\color{blue}{#1}}}

\title[Статическая верификация\ldots]{Статическая верификация рефакторных преобразований структуры программ на основе правил редукции в языке программирования Featherweight Java}

\author[\mbox{А.~Демин}]
{Демин~А.~В.\\
Научный~руководитель д.~ф.-м.~н.,~проф.~Баранов~С.~Н.\\
Соруководитель к.~т.~н.~МакКивер~С.}

\institute[НИУ ИТМО]{Санкт-Петербургский национальный исследовательский университет информационных технологий, механики и оптики}

% \AtBeginSection[] {
%   \begin{frame}<beamer>{}
%     \tableofcontents[currentsection,currentsubsection]
%   \end{frame}
% }
\newcolumntype{C}[1]{>{\centering\let\newline\\\arraybackslash\hspace{0pt}}m{#1}}
\begin{document}

\begin{frame}
  \titlepage
\end{frame}

\section{Описание задачи}
\subsection{Рефакторинг}
\begin{frame}{Рефакторинг}
\begin{itemize}
    \item Рефакторинг -- изменение исходного кода программы, не затрагивающее ее внешнего поведения, с целью облегчить понимание ее работы.
    \item Для проверки используются модульные тесты.
    \begin{itemize}
        \item Не является формальной верификацией.
        \item Проверяют конкретную версию, а не изменения.
    \end{itemize}
\end{itemize}
\end{frame}

\subsection{Featherweight Java}
\begin{frame}{Featherweight Java}
\begin{itemize}
    \item Минимальное ядро для моделирования системы типов языка Java
    \item Определены правила редукции для всех видов выражений ($\rightarrow$)
\end{itemize}
    \begin{align*}
        L &::= class\ C\ extends\ C\ \{\overline{C}\ \overline{f};\ K\ \overline{M}\}\\
        K &::= C(\overline{C}\ \overline{f})\{\ super(\overline{f});\ this.\overline{f} = \overline{f};\ \}\\
        M &::= C\ m(\overline{C}\ \overline{x}\{\ return\ e;\ \}\\
        e &::= x\ |\ e.f\ |\ e.m(\overline{e})\ |\ new\ C(\overline{e})\ |\ (C)e
    \end{align*}
\end{frame}

\subsection{Постановка задачи}
\begin{frame}{Постановка задачи}
\begin{itemize}
    \item Формализовать определение некоторых методов рефакторинга
    \item Доказать его корректность
    \item На основе определений построить алгоритм верификации рефакторингов
\end{itemize}
\end{frame}

\section{Описание применяемого подхода}

\subsection{Эквивалентность}
\begin{frame}{Эквивалентность}
Выражения $\Gamma \vdash e_1$ и $\Gamma \vdash e_2$ назовем эквивалентными по редукции,
если для любых значений переменных из $\Gamma$ либо существует выражение $e'$ такое, что $e_1 \rightarrow^* e'$ и $e_2 \rightarrow^* e'$,
либо ни $e_1$ ни $e_2$ не могут быть редуцированы до нормальной формы.
Эквивалентные по редукции выражения будем обозначать $e_1 \equiv_r e_2$.
\end{frame}

\subsection{Выделение метода}
\begin{frame}{Выделение метода}
Выделение метода -- метод рефакторинга, при котором фрагмент кода, который можно сгруппировать, преобразуется в метод, название которого обозначает его назначение.

Детали:
\begin{itemize}
    \item тело выделенного метода не может содержать прямой рекурсии;
    \item в родительских классах не должен быть определен метод с тем же именем;
    \item в дочерних классах не должен быть определен метод с тем же именем.
\end{itemize}
\end{frame}

\begin{frame}{Редукция по методу}
\begin{itemize}
    \item Необходимо <<раскрыть>> все вызовы выделенного метода
    \item Редукция по методу $m$ класса $C$:
    \begin{itemize}
        \item $\cfrac{\Gamma \vdash this : D \qquad D <: C \qquad \overline{e} \rightarrow_m \overline{e}'}{this.m(\overline{e}) \rightarrow_m [\overline{e}'/\overline{x}]e_m}$
        \item Остальные выражения оставить без изменений
    \end{itemize}
    \item Если $e_1 \rightarrow_{m,C} e_2$, то $e_1 \equiv e_2$
\end{itemize}
\end{frame}

\begin{frame}{Алгоритм верификации}
\begin{enumerate}
    \item Проверить, что выделенный метод не содержит прямой рекурсии.
    \item Проверить, что родительские классы не содержат метод с таким же именем.
    \item Проверить, что дочерние классы не содержат метод с таким же именем.
    \item Для всех методов, содержащих вызовы выделенного метода, проверить, что тело метода до и после рефакторинга находятся в отношении редукции по методу.
\end{enumerate}
\end{frame}

\subsection{Замена условного оператора полиморфизмом}
\begin{frame}{Замена условного оператора полиморфизмом}
До проведения рефакторинга.
\begin{itemize}
    \item Класс $C$, содержащий метод $m$.
    \item $D_1, \ldots, D_n$ -- дочерние классы класса $C$.
    \item Тело метода $m$ -- switch-выражение.
\end{itemize}
\end{frame}

\begin{frame}{Замена условного оператора полиморфизмом}
Нужно установить соответствие между дочерними классами и ветвями switch-выражения.
\begin{itemize}
    \item Вычислить значение всех условий в switch-выражении для всех классов.
    \item Поставить каждому классу в соответствие первую ветвь, условие которое истинно.
    \item Требование: условия должны приводиться к нормальной форме за константное число шагов.
\end{itemize}
\end{frame}

\begin{frame}{Алгоритм верификации}
\begin{enumerate}
    \item Проверить, что тело преобразуемого метода представляет собой switch-выражение.
    \item Проверить, что число ветвей в switch-выражении совпадает с числом дочерних классов.
    \item Для каждого дочернего класса вычислить значения условных выражений всех ветвей switch-выражения.
    \item Каждому дочернему классу поставить в соответствие первую ветвь switch-выражения, условное выражение которой является истинным.
    \item Проверить, что каждая ветвь была поставлена в соответствие только одному классу.
    \item Для каждого дочернего класса проверить, что тело метода после рефакторинга совпадает результатом соответствующей ветви switch-выражения.
\end{enumerate}
\end{frame}

% \lstset{
%     language=Java,
%     basicstyle=\small\ttfamily,
%     frame=single,
%     captionpos=b
% }

\begin{frame}{Спасибо за внимание! Вопросы?}
\end{frame}

\end{document}

