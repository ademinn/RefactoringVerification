\documentclass[hyperref=unicode,graphics=pdflatex,13pt]{beamer}

\mode<presentation>
{
%   \usetheme{Goettingen}
%   \setbeamercovered{invisible}
  \useoutertheme{infolines}
%   \setbeamercovered{invisible}
}

\graphicspath{{pic/}}

\usepackage[utf8]{inputenc}
\usepackage[T2A]{fontenc}
\usepackage[english,russian]{babel}
\usepackage{times}
\usepackage{color, colortbl}
\usepackage{tikz}
\usepackage{calc}
\usepackage{array}
\usepackage{multirow}
\usepackage{listings}
\usetikzlibrary{calc,intersections,through,backgrounds,chains,positioning,fit}

%\usepackage{xcolor}

\newcommand\highlight[1]{{\color{blue}{#1}}}

\title[Статическая верификация\ldots]{Статическая верификация рефакторных преобразований структуры программ на основе правил редукции в языке программирования Featherweight Java}

\author[\mbox{А.~Демин}]
{Демин~А.~В.\\
Научный~руководитель д.~ф.-м.~н.,~проф.~Баранов~С.~Н.\\
Соруководитель к.~т.~н.~МакКивер~С.}

\institute[НИУ ИТМО]{Санкт-Петербургский национальный исследовательский университет информационных технологий, механики и оптики}

% \AtBeginSection[] {
%   \begin{frame}<beamer>{}
%     \tableofcontents[currentsection,currentsubsection]
%   \end{frame}
% }
\newcolumntype{C}[1]{>{\centering\let\newline\\\arraybackslash\hspace{0pt}}m{#1}}
\begin{document}

\begin{frame}
  \titlepage
\end{frame}

\section{Описание задачи}
\subsection{Рефакторинг}
\begin{frame}{Рефакторинг}
\begin{itemize}
    \item Рефакторинг -- изменение исходного кода программы, не затрагивающее ее внешнего поведения, с целью облегчить понимание ее работы.
    \item Для проверки используются модульные тесты.
    \begin{itemize}
        \item Не является формальной верификацией.
        \item Проверяют конкретную версию, а не изменения.
    \end{itemize}
\end{itemize}
\end{frame}

\subsection{Featherweight Java}
\begin{frame}{Featherweight Java}
\begin{itemize}
    \item Минимальное ядро для моделирования системы типов языка Java
    \item Определены правила редукции для всех видов выражений ($\rightarrow$)
\end{itemize}
    \begin{align*}
        L &::= class\ C\ extends\ C\ \{\overline{C}\ \overline{f};\ K\ \overline{M}\}\\
        K &::= C(\overline{C}\ \overline{f})\{\ super(\overline{f});\ this.\overline{f} = \overline{f};\ \}\\
        M &::= C\ m(\overline{C}\ \overline{x}\{\ return\ e;\ \}\\
        e &::= x\ |\ e.f\ |\ e.m(\overline{e})\ |\ new\ C(\overline{e})\ |\ (C)e
    \end{align*}
\end{frame}

\subsection{Постановка задачи}
\begin{frame}{Постановка задачи}
\begin{itemize}
    \item Формализовать определение некоторых методов рефакторинга
    \item Доказать его корректность
    \item На основе определений построить алгоритм верификации рефакторингов
\end{itemize}
\end{frame}

\section{Описание применяемого подхода}

\subsection{Эквивалентность}
\begin{frame}{Строгая вычислимость}
\begin{itemize}
    \item $\overline{e}$ вычислимо, если $\forall e \in \overline{e}\ e$ вычислимо
    \item если $e$ вычислимо, то $e.f$ вычислимо;
    \item если $mbody(m, C) = \overline{x}.e_0$ и вычислимы $e$, $\overline{e}$ и $[\overline{e}/\overline{x}, e/this]e_0$, то $e.m(\overline{e})$ вычислимо;
    \item если $\overline{e}$ вычислимо, то $new\ C(\overline{e})$ вычислимо;
    \item если $e$ вычислимо, то $(C) e$ вычислимо.
\end{itemize}
\end{frame}

\begin{frame}{Эквивалентность}
Выражения $\Gamma \vdash e_1$ и $\Gamma \vdash e_2$ назовем эквивалентными, если для любых значений переменных из $\Gamma$ либо оба выражения не являются строго вычислимыми,
либо оба выражения строго вычислимы и существует выражение $e' : e_1 \rightarrow^* e'$ и $e_2 \rightarrow^* e'$. Эквивалентные выражения будем обозначать $e_1 \equiv e_2$.
\end{frame}

\subsection{Выделение метода}
\begin{frame}{Выделение метода}
Выделение метода -- метод рефакторинга, при котором фрагмент кода, который можно сгруппировать, преобразуется в метод, название которого обозначает его назначение.
\end{frame}

\begin{frame}{Редукция по методу}
\begin{itemize}
    \item Необходимо <<раскрыть>> все вызовы выделенного метода
    \item Тело метода должно содержать все параметры
    \item Редукция по методу $m$ класса $C$:
    \begin{itemize}
        \item $\cfrac{\Gamma \vdash e : C \qquad mbody(m, C) = \overline{x}.e_0}{e.m(\overline{d}) \rightarrow_{m,C} [\overline{d}/\overline{x}, e/this]e_0}$
        \item Остальные выражения оставить без изменений
    \end{itemize}
    \item Если $e_1 \rightarrow_{m,C} e_2$, то $e_1 \equiv e_2$
\end{itemize}
\end{frame}

\begin{frame}{Алгоритм верификации}
\begin{itemize}
    \item Проверить, что тело метода $m$ содержит все параметры.
    \item Найти все методы, содержащие вызов метода $m$.
    \item Для каждого метода $m_i$ из найденных:
    \begin{itemize}
        \item Провести редукцию по методу: $e_i \rightarrow_{m,C} e_i'$.
        \item Проверить, что тело метода до изменений $e_i^0$ и $e_i'$ структурно эквивалентны.
    \end{itemize}
\end{itemize}
\end{frame}

\lstset{
    language=Java,
    basicstyle=\small\ttfamily,
    frame=single,
    captionpos=b
}
\section{Пример применения}
\subsection{Пример применения}
\begin{frame}[fragile]
\frametitle{Пример применения}
\begin{lstlisting}
class Pair extends Object {
    Object fst;
    Object snd;
    
    Pair(Object fst, Object snd) {
        super(); this.fst = fst; this.snd = snd;
    }
}

class PairFactory extends Object {
    PairFactory() { super(); }

    Pair createPair(Object fst, Object snd) {
        return new Pair(fst, snd);
    }
}
\end{lstlisting}
\end{frame}

\begin{frame}[fragile]
\frametitle{Пример применения}
\begin{lstlisting}
class Engine extends Object {
    PairFactory pf;

    Engine(PairFactory pf) {
        super(); this.pf = pf;
    }
    
    Pair getPair(Object fst) {
        return this.pf.createPair(fst, new Object());
    }
}
\end{lstlisting}
\end{frame}

\begin{frame}[fragile]
\frametitle{Пример применения}
После выделения метода $createPairFst$
\begin{lstlisting}
class PairFactory extends Object {
    ...

    Pair createPairFst(Object fst) {
        return this.createPair(fst, new Object()); //OK
        return this.createPair(new Object(), fst); //FAIL
    }
}

class Engine extends Object {
    ...

    Pair getPair(Object fst) {
        return this.pf.createPairFst(fst);
    }
}
\end{lstlisting}
\end{frame}

\begin{frame}{Пример применения}
\begin{itemize}
    \item Исходное тело метода:
    $this.pf.createPair(fst, new\ Object())$
    \item Корректный рефакторинг:
    $this.pf.createPairFst(fst) \rightarrow_{createPairFst,PairFactory} 
    this.pf.createPair(fst, new\ Object())$
    \item Некорректный рефакторинг:
    $this.pf.createPairFst(fst) \rightarrow_{createPairFst,PairFactory} 
    this.pf.createPair(new\ Object(), fst)$
\end{itemize}
\end{frame}

\begin{frame}[fragile]
\frametitle{Пример применения}
Модульный тест
\begin{lstlisting}
class TestEngine extends Test {
    TestResult testGetPair() {
        return this.assertEqual(
            new Engine(new PairFactory()).
                getPair(new Object()),
            new Pair(new Object(), new Object()));
    }
}
\end{lstlisting}
\end{frame}

\begin{frame}{Спасибо за внимание! Вопросы?}
\end{frame}

\end{document}

